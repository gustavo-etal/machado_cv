\documentclass[11pt]{article}
% Full Unicode support for non-ASCII characters
\usepackage[utf8]{inputenc}

% ======================================
% Bibliography
% ======================================
\usepackage[
backend=biber,
maxbibnames=99,
style=numeric,
sorting=ydnt
%citestyle=numeric,
]{biblatex}

%% This will count the references and reverse the order
\addbibresource{citations.bib} %Import the bibliography file
% Count the total number of entries in each reflection
\AtDataInput{%
  \csnumgdef{entrycount:\therefsection}{%
    \csuse{entrycount:\therefsection}+1}}

% Print the label number as the total number of entries in the current ref section, minus the actual labelnumber, plus one
\DeclareFieldFormat{labelnumber}{\mkbibdesc{#1}}    
\newrobustcmd*{\mkbibdesc}[1]{%
  \number\numexpr\csuse{entrycount:\therefsection}+1-#1}

%%%%%%%%%%%%%%%%%%%%%%%%%%%%%%%%%%%%%%%%%%%%%%%%%%%
%% Your information %%%%%%%%%%%%%%%%%%%%%%%%%%%%%%%
%%%%%%%%%%%%%%%%%%%%%%%%%%%%%%%%%%%%%%%%%%%%%%%%%%%

% Useful aliases for main sub-sessions
\newcommand{\NCSU}{North Carolina State University}
\newcommand{\vet}{College of Veterinary Medicine}
\newcommand{\UHEARTH}{Department of Earth Sciences}
\newcommand{\CVM}{Department of Population Health and Pathobiology}

% Identifying information
\newcommand{\Title}{Curriculum Vit\ae}
\newcommand{\FirstName}{Gustavo}
\newcommand{\LastName}{Machado}
\newcommand{\Initials}{}
\newcommand{\MyName}{\FirstName\ \LastName}
\newcommand{\Me}{\textbf{\LastName, \FirstName \Initials }} 

% For citations
\newcommand{\Email}{gmachad@ncsu.edu}
\newcommand{\PersonalWebsite}{machado-lab.github.io}
\newcommand{\LabWebsite}{machado-lab.github.io}
\newcommand{\ORCID}{0000-0001-7552-6144}
\newcommand{\Affiliation}{\CVM \\ \vet \\ \NCSU}
\newcommand{\Address}{
  1060 William Moore Drive, Raleigh NC 27607, USA
}
%%%%%%%%%%%%%%%%%%%%%%%%%%%%%%%%%%%%%%%%%%%%%%%%%%%
% Packages
%%%%%%%%%%%%%%%%%%%%%%%%%%%%%%%%%%%%%%%%%%%%%%%%%%%

% Metadata for the PDF output and control of hyperlinks
\usepackage[colorlinks=true]{hyperref}
\hypersetup{
  pdftitle={\MyName\ - \Title},
  pdfauthor={\MyName},
  linkcolor=blue,
  citecolor=blue,
  filecolor=black,
 urlcolor=MidnightBlue
}
\usepackage[hyphenbreaks]{breakurl}
\usepackage{url} % Evitar enlaces en bibliografía largos & enlazar enlaces de bibliografía
\usepackage{etoolbox}

% Template configuration
%%%%%%%%%%%%%%%%%%%%%%%%%%%%%%%%%%%%%%%%%%%%%%%%%%%%%%%%

% Disable hyphenation
\usepackage[none]{hyphenat}

% Control the font size
\usepackage{anyfontsize}

% Icon fonts (requires using xelatex or luatex)
\usepackage[fixed]{fontawesome5}
\usepackage{academicons}

% Template variables for styling
\newcommand{\TablePad}{\vspace{-0.4cm}}
\newcommand{\SoftwareTitle}[1]{{\bfseries #1}}
\newcommand{\TableTitle}[1]{{\fontsize{12pt}{0}\selectfont \itshape #1}}

% For fancy and multi page tables
\usepackage{tabularx}
\usepackage{ltablex}

% Define a new environment to place all CV entries in a 2-column table.
% Left column are the dates, right column the entries.
\usepackage{environ}
\NewEnviron{EntriesTable}{
\TablePad
\begin{tabularx}{\textwidth}{@{}p{0.12\textwidth}@{\hspace{0.02\textwidth}}p{0.86\textwidth}@{}}
  \BODY
\end{tabularx}
}

% Macros to add links and mark publications
\newcommand{\DOI}[1]{doi:\href{https://doi.org/#1}{#1}}
\newcommand{\DOILink}[1]{\href{https://doi.org/#1}{doi.org/#1}}
\newcommand{\Preprint}[1]{\newline • Preprint: \faFilePdf}
\newcommand{\Youtube}[1]{\newline • Recording: \faYoutube\, \href{https://www.youtube.com/watch?v=#1}{youtube.com/watch?v=#1}}
\newcommand{\GitHub}[1]{\newline • Code: \faGithub\ \href{https://github.com/#1}{#1}}
\newcommand{\Role}[1]{\newline • Role: #1}
\newcommand{\Website}[1]{\newline • Website: \href{https://#1}{#1}}
\newcommand{\Slides}[1]{\newline • Slides: \faTv\ \href{https://#1}{#1}}
\newcommand{\SlidesDOI}[1]{\newline • Slides: \faTv\ \DOILink{#1}}
\newcommand{\PosterDOI}[1]{\newline • Poster: \faImage\ \DOILink{#1}}
\newcommand{\OA}{\aiOpenAccess\enspace}
\newcommand{\Invited}{\newline • \textbf{Invited talk}}

% Macros to set the year and duration on the left column
\newcommand{\Duration}[2]{\fontsize{10pt}{0}\selectfont #1 -- #2}
\newcommand{\Year}[1]{\fontsize{10pt}{0}\selectfont #1}
\newcommand{\Ongoing}{present}
%\newcommand{\Ongoing}{$\ast$}
\newcommand{\Future}{future}
\newcommand{\Review}{in review}
\newcommand{\Accepted}{accepted}
\newcommand{\Appointment}[4]{\textbf{#1} \newline #2 \newline #3 \newline #4}

% Define command to insert month name and year as date
\usepackage{datetime}
\newdateformat{monthyear}{\monthname[\THEMONTH], \THEYEAR}

% Set the page margins
\usepackage[a4paper,margin=1.5cm,includehead,headsep=5mm]{geometry}

% To get the total page numbers (\pageref{LastPage})
\usepackage{lastpage}

% No indentation
\setlength\parindent{0cm}

% Increase the line spacing
\renewcommand{\baselinestretch}{1.1}
% and the spacing between rows in tables
\renewcommand{\arraystretch}{1.5}

% Remove space between items in itemize and enumerate
\usepackage{enumitem}
\setlist{nosep}

% Use custom colors
\usepackage[usenames,dvipsnames]{xcolor}

% Set fonts. Requires compilation with xelatex
\usepackage{fontspec}  % required to make older xelatex compile with UTF8

% Configure the font style for sections
\usepackage{sectsty}
\sectionfont{\vspace{0.5cm}\bfseries\fontsize{12pt}{0}\selectfont\uppercase}
\subsectionfont{\vspace{0.2cm}\mdseries\fontsize{12pt}{0}\selectfont\uppercase}

% Set the spacing for sections
%\usepackage{titlesec}
%\titlespacing{\section}{0pt}{0cm}{0.3cm}
%\titlespacing{\subsection}{0pt}{0.3cm}{0.3cm}

% Disable the number of sections. Use this instead of "section*" so that the sections still
% appear as PDF bookmarks. Otherwise, I would have to add the table of contents entries
% manually.
\makeatletter
\renewcommand{\@seccntformat}[1]{}
\makeatother

% Set fancy headers
\usepackage{fancyhdr}
\pagestyle{fancy}
\fancyhf{}
\chead{
  \fontsize{10pt}{12pt}\selectfont
  \MyName
  \hspace{0.2cm} -- \hspace{0.2cm}
  \Title
  \hspace{0.2cm} -- \hspace{0.2cm}
  \monthyear\today
}
\rhead{\fontsize{10pt}{0}\selectfont \thepage/\pageref*{LastPage}}
\renewcommand{\headrulewidth}{0pt}

%%%%%%%%%%%%%%%%%%%%%%%%%%%%%%%%%%%%%%%%%%%%%%%%%%%%%%%%%

\begin{document}
% No header for the first page
\thispagestyle{empty}

%%%%%%%%%%%%%%%%%%%%%%%%%%%%%%%%%%%%%%%%%%%%%%%%%%%%%%%%%
% HEADER
{\fontsize{22pt}{0}\selectfont\MyName}\\[-0.1cm]
\rule{\textwidth}{0.2pt}
\begin{minipage}[t]{0.595\textwidth}
  \Affiliation
  \\
  \Address
\end{minipage}
\begin{minipage}[t]{0.405\textwidth}
  \begin{flushright}
  Last updated: \monthyear\today
  \\
    ORCID: \href{https://orcid.org/\ORCID}{\ORCID}
    \\
    email: \href{mailto:\Email}{\Email}
    \\
    Lab web site: \href{https://\LabWebsite}{\LabWebsite}
  \end{flushright}
\end{minipage}

%%%%%%%%%%%%%%%%%%%%%%%%%%%%%%%%%%%%%%%%%%%%%%%%%%%%%%%%%
\section{Short bio}

I am an Associate Professor of Transboundary disease epidemiology in the Population Health and Pathobiology department, and is also affiliated with the
Biomathematics graduate program and with the Center for Geospatial Analytics at NCSU.
I am an infectious disease modeler with expertise in food animal production epidemiology and developing disease control strategies at multiple scales to reduce the burden of endemic and emerging diseases. I have worked on several aspects of disease modeling approaches, including the development of mathematical models fitted to actual population, animal, and vehicle movement data and modeling the epidemiology and disease spread dynamics of several animal diseases, including African swine fever, foot-and-mouth disease, porcine reproductive and respiratory syndrome virus, and porcine epidemic diarrhea virus. I have experience in the regulatory aspects of disease control work nationally and internationally, including Brazil. At NC State University, I teach courses on the epidemiology of infectious diseases. My research laboratory specializes in mathematical models of livestock diseases and develops novel strategies for their control and elimination. We engage with and provide training for government and industry stakeholders and academics nationally and internationally.

\section{Professional Appointments}
\begin{EntriesTable}
  \Duration{2023}{\Ongoing}  &
  \Appointment{Associate Professor}{\CVM}{\vet}{\NCSU, USA}
  \\
  \Duration{2018}{2023}  &
  \Appointment{Assistant Professor}{\CVM}{\vet}{\NCSU, USA}
  \\
  \Duration{2016}{2017}  &
  \Appointment{Postdoctoral researcher}{Department of Veterinary Population Medicine }{College of Veterinary Medicine }{University of Minnesota, USA}
  \\
  \Duration{2016}{2016}   &
  \Appointment{Assistant Professor}{Department of Preventive Veterinary Medicine}{Federal University of Rio Grande do Sul, Brazil}
\end{EntriesTable}
%%%%%%%%%%%%%%%%%%%%%%%%%%%%%%%%%%%%%%%%%%%%%%%%%%%%%%%%%
\section{Education}

\begin{EntriesTable}
  \Duration{2013}{2016}  &
  \textbf{Ph.D. in Veterinary Epidemiology}, Federal University of  Rio Grande do Sul, Brazil
  \\
  \Duration{2011}{2013}  &
  \textbf{MVSc in Veterinary Epidemiology}, Federal University of  Rio Grande do Sul, Brazil
  \\
  \Duration{2003}{2010}  &
  \textbf{DVM Veterinary Medicine}, Federal University of  Santa Maria, Brazil
\end{EntriesTable}
%%%%%%%%%%%%%%%%%%%%%%%%%%%%%%%%%%%%%%%%%%%%%%%%%%%%%%%%%
\section{Funding}

\textbf{Summary of Funding:} 31 Total Funded Projects; Total Funding of \$7.3 million including PI of
nationally competitive grants, including national-level competition, six USDA-NIFA grants, six USDA-APHIS-NADPRP, one Foundation for Food \& Agriculture Research (FFAR) grant, four Swine Health Information Center (SHIC) grants, two international Fundo de Desenvolvimento e Defesa Sanitaria Animal (FUNDESA RS) grants; Total Funding as a PI: \$3.7 million.

\vspace{0.5cm}

Ongoing research support \textbf{(14)}

\begin{EntriesTable}
  \Duration{2023}{2025}  &
   \textbf{PI}: \Me\
`Extending a Between-Farm African Swine Fever Transmission Model to
Estimate the Necessary Number of Sample Collectors in a Highly Swine
Dense Region'' \textbf {Funded by:}  Animal and Plant Health Inspection Service (USDA-APHIS)-National Animal Disease Preparedness and Response Program (NADPRP)``
  \textit{\textbf{Amount \$244,274}}.  
  \\
  \Duration{2023}{2025}  &
   \textbf{PI}: \Me\
`Standardized On-Farm Biosecurity Poultry Producers Plans in a User-Friendly Management Tool for Pennsylvania'' \textbf {Funded by:}  Animal and Plant Health Inspection Service (USDA-APHIS)-National Animal Disease Preparedness and Response Program (NADPRP)``
  \textit{\textbf{Amount \$537,193}}.  
  \\
\Duration{2023}{2025}  &
  PI: Ferreira, \textbf{co-PI}: \Me\
`Evaluation of Decontamination Protocols and Vehicle Movement to
Mitigate the Transmission Risk of PED Virus as a Proxy for FAD'' \textbf {Funded by:}  Animal and Plant Health Inspection Service (USDA-APHIS)-National Animal Disease Preparedness and Response Program (NADPRP)``
  \textit{\textbf{Amount \$431,064}}.  
  \\
  \Duration{2022}{2024}  &
   \textbf{PI}: \Me\
``Rerouting between-farm transportation vehicle movements to minimize the dissemination of endemic and emerging diseases in North America'' \textbf {Funded by:} Swine Health Information Center (SHIC)``
  \textit{\textbf{Amount \$131,671}}.  
  \\
\Duration{2022}{2024}  &
  \textbf{PI}: \Me\
  ``Major Enhancement of U.S. swine industry preparedness and responses to large-scale infectious foreign animal diseases through harmonizing biosecurity plans and advancing interpretable machine learning'' \textbf {Funded by:} Foundation for Food and Agriculture Research (FFAR)-New Innovator Fellowship``
  \textit{\textbf{Amount \$479,987}}.
   \\
   \Duration{2022}{2024}  &
  \textbf{PI}: \Me\
  ``Enhancing U.S. Swine Industry
Readiness to Foreign Animal Disease at
All Production Levels in Missouri By
Combining Standardized On-Farm
Biosecurity Plans and Animal
Movement in a Data Management
Tool'' \textbf {Funded by:}
  Animal and Plant Health Inspection Service (USDA-APHIS)-National Animal Disease Preparedness and Response Program (NADPRP)``
  \textit{\textbf{Amount \$271,730}}.
  \\
   \Duration{2022}{2024}  &
  \textbf{PI}: \Me\
  ``Descriptive Analysis of Multiple Swine
Movement Networks and The
Development of a Network Model to
Estimate the Impacts of the Movement
Restrictions Under the National African
Swine Fever Response Plan'' \textbf {Funded by:}
  Animal and Plant Health Inspection Service (USDA-APHIS)-National Animal Disease Preparedness and Response Program (NADPRP)``
  \textit{\textbf{Amount \$312,012}}.
  \\
  \Duration{2022}{2024}  &
  \textbf{PI}: \Me\
  ``Developing a Tool for Standardization
for Cataloging, Reviewing, and approving of Secure Beef Supply Plans
of Producers on Different Types of
Operations in Kansas'' \textbf {Funded by:}
  Animal and Plant Health Inspection Service (USDA-APHIS)-National Animal Disease Preparedness and Response Program (NADPRP)``
  \textit{\textbf{Amount \$405,110}}.
    \\
    \Duration{2022} {2024} &
  \textbf{PI}: \Me\
  ``Develop the Epidemiological Framework Necessary to Reconstruct Vehicle Movement Network'' \textbf {Funded by:}
  Swine Health Information Center (SHIC)``
  \textit{\textbf{Amount \$125,000}}.
  \\
  \Duration{2020}{2024}  &
  \textbf{PI}: \Me\
  ``Near-real time spatiotemporal resource allocation to improve swine health'' \textbf {Funded by:} USDA-AFRI Foundational and Applied Science``
  \textit{\textbf{Amount \$500,000}}. Award ID: {20206702132462}.
  \\
  \Duration{2021}{2025}  &
  \textbf{PI}: \Me\
  ``Risk-guided surveillance: analysis of complex networks, strengthening of the sentinel program, population dynamics and identification of areas at risk for the spread of endemic and exotic diseases'' \textbf {Funded by:} FUNDESA-Fundo de Desenvolvimento e Defesa Animal``
  \textit{\textbf{Amount \$231,000}}. Award ID: {2021-0715}.
  \\
  \Duration{2022}{2025} &
    \textbf{Multi-PI}: Tobias, Kaeser and \Me\ | \emph{(Machado, G., \$340,000)}
  ``Predict and protect against PRRSV (Preproprrsv): Combine PRRSV forecasting technology with vaccine efficacy prediction to prevent PRRSV outbreak'' \textbf {Funded by:}
  USDA-AFRI Foundational and Applied Science-IDEAS``
  \textit{\textbf{Amount \$1,000,000}}.
   \\
  \Duration{2022}{2025}  & PI Linhares, Daniel,
  \textbf{co-PI}: \Me\ | \emph{(Sub award for Machado, G., \$289,000)}
  ``Integrating data streams for causal inference and forecasting application to foster precision swine health'' \textbf {Funded by:}
  USDA-AFRI Foundational and Applied Science-IDEAS``
  \textit{\textbf{Amount \$1,000,000}}.
  \\
\Duration{2023}{2026}  & PI Gustavo, Silva,
  \textbf{co-PI}: \Me\ 
``Ongoing automated surveillance of swine livestock productivity data to early detect PRRSV outbreaks'' \textbf {Funded by:}
  USDA-AFRI Foundational and Applied Science-CARE``
  \textit{\textbf{Amount \$300,000}}.
\end{EntriesTable}

%%%%%%%%%%%%%%%%%%%%%%%%%%%%%%%%%%%%%%%%%%%%%%%%%%%%%%%%%%%%%%%%
Completed research support \textbf{(16)}
\begin{EntriesTable}
    \Duration{2022} {2022} &
  \textbf{PI}: \Me\
  ``National Pork Board API Development Agreement'' \textbf {Funded by:}
  National Pork Board``
  \textit{\textbf{Amount \$30,000}}. Award ID.
  \\
\Duration{2022}{2023} & PI Pairis-Garcia, Monique,
  \textbf{co-PI}: \Me\ | \emph{(Sub award for Machado, G., \$89,000}
  ``Refining sample size recommendations for PQA Plus® and CSIA audit tool'' \textbf {Funded by:}
  National Pork Board``
  \textit{\textbf{Amount \$179,500}}.
  \\
  \Duration{2021}{2023}  &
  \textbf{PI}: \Me\
  ``Combining standardized on-farm biosecurity plans with animal movement data in a user-friendly rapid access biosecurity management tool: a multi state study'' \textbf {Funded by:}
  Animal and Plant Health Inspection Service (USDA-APHIS)-National Animal Disease Preparedness and Response Program (NADPRP)``
  \textit{\textbf{Amount \$182,301}}. Award ID: {APP-15216}.
  \\
  \Duration{2019}{2023}  &
  \textbf{PI}: \Me\
  ``Assessing biosecurity vulnerabilities to predict the risk of new Porcine Reproductive and Respiratory Syndrome outbreaks'' \textbf {Funded by:} USDA-AFRI Foundational and Applied Science``
  \textit{\textbf{Amount \$300,000}}. Award ID: {20196800829910}.
  \\
\Duration{2021}{2023}  &
  \textbf{PI}: \Me\
  ``Development of a machine-learning framework to identify risk factors for COVID-19 infection in swine caretakers and estimate the chances of new COVID-19 waves`` \textbf {Funded by:} Grant Competition (UGPN)``
  \textit{\textbf{Amount \$30,000}}. Award ID: {14}.
\\
  \Duration{2020}{2021}  &
  \textbf{PI}: \Me\
  ``Dynamic transmission modeling-the role of feed, feed ingredients in swine disease transmission'' \textbf {Funded by:} Fats and Proteins Research Foundation``
  \textit{\textbf{Amount \$76,000}}. Award ID: {2021-0715}.
  \\
  \Duration{2019}{2020} &
  \textbf{PI}: \Me\
  ``High-resolution dynamic risk mapping to guide timely disease interventions- Swine health information center'' \textbf {Funded by:} SHIC-Swine Health Information 
  \textit{\textbf{Amount \$89,987}}. Award ID: {17-141}.
  \\
  \Duration{2019}{2020}  &
  \textbf{PI}: \Me\
  ``Network analysis to guide active surveillance'' \textbf {Funded by:} FUNDESA-Fundo de Desenvolvimento e Defesa Animal``
  \textit{\textbf{Amount \$22,696}}. Award ID: {2019-049}.
 \\
  \Duration{2019}{2020}  &
  {PI}: Linhares, \textbf{Co-PI} \Me\
  ``Biosecurity screening tool to identify breeding herds’ risk to PRRS outbreak using a short survey Swine health information center'' \textbf {Funded by:} SHIC-Swine Health Information Center``
  \textit{\textbf{Amount \$75,201}}. Award ID: {2019-2601}.
  \\
  \Duration{2019}{2020}  &
  {PI}: Goss, \textbf{Co-PI} \Me\
  ``Distribution and activity of Pythium insidiosum in the Chincoteague National Wildlife Refuge'' \textbf {Funded by:} U.S. Fish and Wildlife Service``
  \textit{\textbf{Amount \$11,292}}. Award ID: {2019-2012}.
  \\
 \Duration{2018}{2020}  &
  {PI}: VanderWall, \textbf{Co-PI} \Me\
  ``Forecasting the spread of endemic viruses of swine in the United States'' \textbf {Funded by:} USDA-AFRI Foundational and Applied Science``
  \textit{\textbf{Amount \$300,000}}. Award ID: {2018-68008-27890}.
  \\
    \Duration{2019}{2020}  &
  \textbf{PI}: \Me\
  ``Development of a dissemination platform for temporal, spatial and phylogenetic analysis of Avian Infectious Bronchitis Virus sequences`` \textbf {Funded by:} NCSU Research and Innovation Seed Funding Program``
  \textit{\textbf{Amount \$18,750}}. Award ID: {2019-2366}.
  \\
 \Duration{2019}{2020}  &
  \textbf{PI}: \Me\
  ``Identification of key factors enabling PRRSv spatial distribution and the importance of new virus variant introduction on the incidence of PRRSv in North Carolina`` \textbf {Funded by:} NCSU College of Veterinary Medicine``
  \textit{\textbf{Amount \$25,000}}. Award ID: {14}.
  \\
   \Duration{2019}{2020}  &
  \textbf{PI}: \Me\
  ``Use of swine movement information to improve risk-based surveillance`` \textbf {Funded by:} CVM- Global Health grant program``
  \textit{\textbf{Amount \$19,792}}. Award ID: {14}.
 \\
  \Year{2021}  &
  \textbf{PI}: \Me\
  ``Foot and mouth disease in Brazil-cattle, swine, small ruminants and poultry (Aya Omar-DVM Student)'' \textbf {Funded by:} Triangle Community Foundation``
  \textit{\textbf{Amount \$5,000}}. Award ID: {3425}.
\\
    \Duration{2019}{2022}  &
  \textbf{PI}: \Me\
  ``Optimizing control interventions for Visceral Leishmaniasis in multiple settings`` \textbf {Funded by:} Grant Competition (UGPN)``
  \textit{\textbf{Amount \$30,000}}. Award ID: {22}.
\end{EntriesTable}

%%%%%%%%%%%%%%%%%%%%%%%%%%%%%%%%%%%%%%%%%%%%%%%%%%%%%%%%%%%%%%%%%
Consulting roles
\begin{EntriesTable}
  \Duration{2023}{2025}  &
  PI: Michael Sanderson (Kansas State University)\textbf{Consultant}: \Me\
  ``Longitudinal Surveillance and Biosecurity Practice in an FMD Outbreak'' \textbf {Funded by:} Animal and Plant Health Inspection Service (USDA-APHIS)-National
Animal Disease Preparedness and Response Program (NADPRP)``
  \textit{\textbf{Amount \$192,047}}.
\end{EntriesTable}

%%%%%%%%%%%%%%%%%%%%%%%%%%%%%%%%%%%%%%%%%%%%%%%%%%%%%%%%%%%%%
\section{Awards \& Honors}

\begin{EntriesTable}
   \Duration{2022}{2024}  &
  Goodnight Early Career Innovators Award, \textit{\textbf{Amount \$66,000}}
  \\
   \Duration{2022}{2024}  &
  Foundation for Food and Agriculture Research (FFAR)-New Innovator Fellowship, \textit{\textbf{Amount \$479,987}}
  \\
 \Duration{2013}{2016}  &
  Brazilian Ministry of Education CAPES \textbf{PhD Research Scholarship, 3 years funding}
  \\
  \Duration{2011}{2013}  &
  Brazilian Ministry of Education CAPES \textbf{Masters Research Scholarship, 3 years funding}
  \\
  \Duration{2006}{2007}  &
 MAST International | MAST International at University of Minnesota (US) \textbf{International exchange student}
  \\
  \Duration{2009}{2011}  &
  Brazilian Ministry of Education CAPES \textbf{DVM Research
  Scholarship}
\end{EntriesTable}
%%%%%%%%%%%%%%%%%%%%%%%%%%%%%%%%%%%%%%%%%%%%%%%%%%%%%%%%%%%%%
% ======================================
% PUBLICATIONS
% ======================================

\section{Publications}
\textbf{Summary of Publications:} Total Publications in Refereed Journals or Accepted for
Publication: 162; Total Citations: 2765 (Google Scholar); h-index: 28 (Google Scholar); * Denotes Student or Researcher Advisee.

\subsection{Preprint}
\begin{EntriesTable}

\Year{2024}  &
Cardenas, N.C., Valencio, A., Sanchez, F., O’Hara, K.C., {\textbf{Machado, Gustavo*.}}\
  Analyzing the intrastate and interstate swine movement network in the United States.
  %\emph{arXiv}.
  \Preprint{}\href{https://www.biorxiv.org/content/10.1101/2024.01.25.576551v1}{Link}
  %\GitHub{}
  \\
\Year{2023}  &
Galvis, O.J, {\textbf{Machado, Gustavo*.}}\
  The role of vehicle movement in swine disease dissemination: novel method accounting for pathogen stability and vehicle cleaning effectiveness uncertainties.
  %\emph{arXiv}.
  \Preprint{}\href{https://arxiv.org/abs/2212.07466}{Link}
  %\GitHub{}
  \\
\Year{2022}  &
 Cardenas, N.C., Lopes, F.P.N., {\textbf{Machado, Gustavo*.}}\
  Modeling foot-and-mouth disease dissemination in Brazil and evaluating the effectiveness of control measures.
  %\emph{arXiv}.
  \Preprint{}\href{https://www.biorxiv.org/content/10.1101/2022.06.14.496159v2}{Link}
  %\GitHub{}
  \\

\Year{2022}  &
da Costa, J.M.N., Cobellini, L.G., Cardenas, N.C., Groff, F.H.S., {\textbf{Machado, Gustavo*.}}\ Assessing epidemiological parameters and dissemination characteristics of the 2000 and 2001 foot-and-mouth disease outbreaks in Rio Grande do Sul, Brazil.
  %\emph{arXiv}.
  \Preprint{}\href{https://www.biorxiv.org/content/10.1101/2022.05.22.492961v1?rss=1}{Link}
  %\GitHub{}
\end{EntriesTable}

%%%%%%%%%%%%%%%%%%%%%%%%%%%%%%%%%%%%%%%%%%%%%%%%%%%%%%%
\subsection{Book and Book Chapters}
\begin{EntriesTable}

\Year{2023}  &
  \textbf{Machado, Gustavo.}, Jason O. A. Galvis, Allyson Freeman, Felipe Sanchez, Xena Hong, Abagael Sykes, Christian Fleming, et al. 2023. “The Rapid Access Biosecurity (RAB) App™ Handbook.” OSF Preprints. January 13. doi:10.17605/OSF.IO/Z5WBJ.
  %\emph{arXiv}.
  \Preprint{}\href{https://osf.io/p5uwq/}{Link}
  %\GitHub{}
\end{EntriesTable}

\nocite{*}

\printbibliography[title=PEER-REVIEWED]

%%%%%%%%%%%%%%%%%%%%%%%%%%%%%%%%%%%%%%%%%%%%%%%%%%%%%%%%%
\section{Open-source Software}

\begin{EntriesTable}
  \Duration{2021}{\Ongoing} &
  \textbf{MrIML}
  \newline
  Multivariate (multi-response) interpretable machine learning
  \Role{Creator and core developer}
  \GitHub{nfj1380/mrIML/}
  \Website{nfj1380.github.io/mrIML/index.html}
  \\
  \Duration{2020}{\Ongoing} &
  \textbf{Research laboratory Jekyll site (Ruby)}
  \newline
  Research laboratory
  \Role{Creator, main developer, project leadership}
  \Website{machado-lab.github.io}
\end{EntriesTable}

%%%%%%%%%%%%%%%%%%%%%%%%%%%%%%%%%%%%%%%%%%%%%%%%%%%%%
\section{Supervision and mentoring}

\subsection{Research Scholars}

\begin{EntriesTable}

\Duration{2023}{\Ongoing} &
  Dr. Jason Galvis
  \newline
  NC State University
\\

\Duration{2023}{\Ongoing} &
  Dr. Nicolas Cardenas
  \newline
  NC State University
\end{EntriesTable}

%%%%%%%%%%%%%%%%%%%%%%%%%%%%%%%%%%%%%%%%%%%%%%%%%%%%%`
\subsection{Postdoctoral researchers}

\begin{EntriesTable}

\Duration{2023}{\Ongoing}  &
  Dr. Aniruddha Deka
  \newline
  NC State University
\\

\Duration{2021}{2023}  &
  Dr. Nicolas Cardenas
  \newline
  NC State University
\\
\Duration{2020}{2023}  &
  Dr. Jason Galvis
  \newline
  NC State University
\\
\Duration{2022}{2023}  &
  Dr. Arthur Valencio
  \newline
  NC State University
\\
\Duration{2018}{2020}  &
  Dr. Manuel Jara
  \newline
  NC State University
\end{EntriesTable}

\subsection{P.h.D}

\begin{EntriesTable}
\Duration{2021}{\Ongoing}  &
  Abagael Sykes (Chair)
  \newline
  NC State University-CBS
\\
\Duration{2021}{\Ongoing}  &
  Felipe Sanchez (Chair)
  \newline
  NC State University-CGA
\\
\Duration{2023}{\Ongoing}  &
  Faith Kennedy (Chair)
  \newline
  NC State University-CGA
\\
\Duration{2024}{\Ongoing}  &
  Christian Fleming (Chair)
  \newline
  NC State University-CGA
\\
\Duration{2021}{2022}  &
  Parker Trostle (co-chair)
  \newline
  NC State University-STA
  \\
\Duration{2021}{2022}  &
  João Marcos N. da Costa (co-chair)
  \newline
  Universidade Federal do Rio Grande do Sul (International)
  \\
  \Duration{2021}{2023}  &
  Anna Isabel Suñé (co-chair)
  \newline
  Universidade Federal do Rio Grande do Sul (International)
\end{EntriesTable}

\subsection{Master}

\begin{EntriesTable}
\Duration{2021}{2023}  &
  Christian Fleming (Chair)
  \newline
    NC State University-CGA
\\
\Duration{2021}{2022}  &
  Cameron Ellington (co-chair)
  \newline
  NC State University-resident
\\
\Duration{2020}{2020} &
  Kelsey Mills (Chair)
  \newline
  NC State University-CGA
\\
  \Duration{2020}{2020} &
 Heather Paxson  (advisor)
  \newline
  NC State University-CGA
\\
\Duration{2020}{2020} &
 Patrícia Warzensaky Gottardo Balestrin (Co-chair)
  \newline
  UDESC-Brazil
\end{EntriesTable}

\subsection{DVM-students}

\begin{EntriesTable}
\Duration{2021}{2022}  &
  Alyssa Valentine
    \newline
  NC State University
  \\
\Duration{2021}{\Ongoing}  &
  Elizabeth Farren Walsh
    \newline
  NC State University
  \\
\Duration{2020}{2020}  &
  Aya Omar
    \newline
  NC State University
  \\
\Duration{2018}{2021}  &
  Stephanie Krasteva
    \newline
  NC State University
  \\
\Duration{2019}{2020}  &
  Jamie Madigan
    \newline
  NC State University
  \\
\Duration{2019}{2020}  &
  Jack Lee Miller
    \newline
  NC State University
\end{EntriesTable}


\subsection{Undergraduate}

\begin{EntriesTable}
\Duration{2020}{2022}  &
  Allyson Freeman
    \newline
  NC State University
  \\
\Duration{2021}{2022}  &
  Madison Joyce  
  \newline
  NC State University
  \\
\Duration{2021}{2021}  &
  Alyssa White
    \newline
  NC State University
  \\
\Duration{2021}{2021}   &
  Grace Winesett
    \newline
  NC State University
  \\
\Duration{2019}{2020}  &
  Victoria J. Reynolds
    \newline
  NC State University
  \\
\Duration{2020}{2020}   &
  Brittany Lee
    \newline
  NC State University
  \\
\Duration{2018}{2018} &
  Bridget Knapp
    \newline
  NC State University
\end{EntriesTable}

%%%%%%%%%%%%%
\subsection{MS and PhD thesis committees}

\begin{EntriesTable}
\Duration{2023}{current}  &
  Lindsey Britton,  Animal Science M.S. Program
  \newline
  NC State University
  \\
\Duration{2018}{2020}  &
  Dr. Trevor Farthing, CBS – Infectious Diseases Epidemiology
  \newline
  NC State University
\\
\Duration{2020}{2022}  &
  Dr. Cameron Ellington, MS Poultry Epidemiology
  \newline
  NC State University
\end{EntriesTable}

%%%%%%%%%%%%%%%%%%%%%%%%%%%%%%%%%%%%%%%%%%%%%%%%%%%%
\section{Teaching}

\subsection{DVM-level}

\begin{EntriesTable}
  \Duration{2022}{\Ongoing}  &
 VMP 973 Special Topics in Epidemiology.
  \\
  \Duration{2022}{\Ongoing}  &
  VMP 993 Extramural in Epidemiology, Public Health, and Public Policy.
  \\
  \Duration{2021}{\Ongoing}  &
  VMP 979 Clinical Epidemiology.
  \\
  \Duration{2021}{\Ongoing}  &
  VMP 971 Food Animal Diagnostics for Disease Diagnosis, Control, and Population Surveillance.
  \\
  \Duration{2020}{2022}  &
  VMP 979 Epidemiology, Public Health, and Public Policy.
  \\
  \Duration{2020}{2022}  &
  VMP-991-151 Trade and Globalization (Guest lecturer).
  \\
  \Duration{2018}{\Ongoing}  &
  VMP 991 Special Topics in PHP, Transboundary Disease, and Spatial Epidemiology.
  \\
  \Duration{2019}{\Ongoing}  &
  VMP 904 Swine Industry (Guest lecturer).
  \\
 \Duration{2018}{2018}  &
  VMP 945 Epidemiology and Public Health.
\end{EntriesTable}

\subsection{Graduate-level}

\begin{EntriesTable}
  \Duration{2021}{\Ongoing}  &
  CBS 595 Epidemiology I.
  \\
  \Duration{2021}{\Ongoing}  &
  CBS 775 (001) Spring 2021 Designing population-based research.
  \\
  \Duration{2019}{\Ongoing}  &
  CBS 650/565 Fundamentals of biomedical sciences.
  \\
  \Duration{2019}{2022}  &
  CBS 595 Independent studies in epidemiology.
  \\
  \Duration{2019}{\Ongoing}  &
  VMP 904 Swine Industry (Guest lecturer).
\end{EntriesTable}

\subsection{Invited presentations}

\begin{EntriesTable}

\Year{2024}  &
\Me\ Rerouting Between-farm Transportation Vehicle Movements to
Minimize the Dissemination of Endemic and Emerging Diseases in
North America
  \emph{Advanced Self-Powered Systems of Integrated Sensors and Technologies (ASSIST) Center ERC},
  Raleigh, NC.
  \\
\Year{2023}  &
\Me\ Rerouting Between-farm Transportation Vehicle Movements to Minimize the Dissemination of Endemic and Emerging Diseases in North America
  \emph{2023 NAPRRS/NC229: International Conference of Swine Viral Diseases},
  Chicago, IL.
  \\
  
\Year{2023}  &
\Me\ How on-farm biosecurity, pig, and vehicle movement explain between-farm disease dissemination?
  \emph{COST (European Cooperation in Science and Technology) BETTER webinar},
  Online.
  \\

\Year{2023}  &
\Me\ Advanced precision technologies for enhancing swine disease prevention,
  \emph{Food animal summit},
  Raleigh, NC.
  \\

\Year{2023}  &
\Me\ The role of different truck movements and effectiveness of vehicle cleaning and disinfection in swine disease spread
  \emph{Western Canadian Association of Swine Veterinarians, 2023 Conference},
  Saskatoon, Saskatchewan-Canada.
  \\
  
\Year{2023}  &
\Me\ Rapid access to on-farm biosecurity and between-farm animal movements to expedite the U.S. swine industry's response to and recovery from large-scale infectious diseases
  \emph{Western Canadian Association of Swine Veterinarians, 2023 Conference},
  Saskatoon, Saskatchewan-Canada.
  \\
  
\Year{2023}  &
\Me\ An introduction to RABapp\textsuperscript{\texttrademark}
  \emph{Southern Animal Health Association (NAHA)},
  Charlottesville, VA.
  \\
  
\Year{2023}  &
  \Me\ Enhancement of U.S. swine biosecurity \& an ASF mathematical transmission model to simulate control options,
  \emph{Swine Innovation Forum},
  Goldsboro, NC.
  \\
  
\Year{2023}  &
  \Me\ Modelagem matemática da disseminação da febre aftosa
no Brasil e avaliação da eficácia das medidas de controle,
  \emph{Sociedad Iberoamericana de Epidemiología Veterinaria y Medicina Preventiva},
  Sociedad Iberoamericana de Epidemiologia Veterinaria y Medicina Preventiva (SIEVMP), Online.
  \\
  
\Year{2023}  &
  \Me\ Gentle introduction to RABapp\textsuperscript{\texttrademark},
  \emph{Northeast US Animal Health Association (NEUSAHA)},
  Portsmouth, NH.
  \\
  
\Year{2023}  &
  \Me\ Major Enhancement of U.S. Swine Industry Biosecurity,
  \emph{Cross-Border Threat Screening and Supply Chain Defense (CBTS) Center of Excellence (COE), Distinguished Speaker Series},
  Online, Texas.
  \\
  

\Year{2023}  &
  \Me Enhancement of U.S. swine industry preparedness and responses to endemic and large-scale infectious FAD through harmonizing biosecurity plans,
  \emph{The Graduate Program in Veterinary Medicine, PPGMV},
  Online, Brazil.
  \\
  

\Year{2022}  &
  \Me\ Major Enhancement of U.S. swine industry biosecurity: what does this mean for a production system, and how on-farm biosecurity, pig movement data, can help explain the between-farm disease dissemination?,
  \emph{2022 NAPRRS/NC229: International Conference of Swine Viral Diseases},
  Chicago, IL.
  \\


\Year{2022}  &
  \Me\ U.S. swine industry preparedness for emerging diseases,
  \emph{GIS Week Lightning Talk(NCSU)},
  Raleigh, NC.
  \\


\Year{2022}  &
  \Me\ PEDV and PRRSV spread dynamics: how to minimize between-farm dissemination,
  \emph{NC Veterinary Conference},
  Raleigh, NC.
  \\

\Year{2022}  &
  \Me\ Artificial intelligence predictions and explanations,
  \emph{Cochran Fellowship Training},
  Sait Paul, MN.
  \\
  
\Year{2022}  &
  \Me\
Major enhancement of U.S. swine industry preparedness and responses to FADs harmonizing biosecurity and advancing disease transmission modeling,
  \emph{Allen D. Leman Swine Conference-Greeks to Greeks},
  Sait Paul, MN.
  \\
  
  
\Year{2022}  &
  \Me\
Advanced disease transmission modeling to enhance U.S. swine industry preparedness for emerging diseases,
  \emph{Advancing sustainable management of pests and pathogens to enhance agricultural productivity-Pests and Pathogens Research Showcase},
  Online.
  \\
  

\Year{2022}  &
  \Me\
Evaluation of the effectiveness of foot-and-mouth disease control measures using modeling in the state of Rio Grande do Sul, Brazil,
  \emph{Regular meeting of the South American commission for the fight against foot-and-mouth (FMD) disease-COSALFA-49},
  Online.
  \\
  
  
\Year{2022}  &
  \Me\
Major Enhancement of U.S. swine industry preparedness and responses to ASF through harmonizing biosecurity plans and advancing disease transmission modeling,
  \emph{NC Swine Veterinarians Meeting},
  Kenansville, NC.
  \\
  
\Year{2022}  &
  \Me\
Modelling and assessing additional transmission routes for
PRRSV: vehicle movements and feed ingredients,
  \emph{Morrison Forum for Advancing Swine Production Medicine},
  Mankato, MN.
  \\
\Year{2022}  &
  \Me\
Modelling transmission dynamics of routes for porcine reproductive and respiratory syndrome virus: vehicle movements and feed ingredients,
  \emph{Swine Debate Group at Iowa State University},
  Online.
  \\
\Year{2022}  &
  \Me\
Introduction to RABapp\textsuperscript{\texttrademark},
  \emph{US SHIP Site Biosecurity Plans Tier 1 Working Group},
  Online.
  \\
\Year{2022}  &
  \Me\
  Modeling for Infectious Bronchitis and Salmonella,
  \emph{Sensor Data and Analytics for Poultry Health, Welfare, and Food Safety, 2022 ACPV Workshop},
  Vancouver, BC Canada.
  \\
\Year{2021}  &
  \Me\
  Modeling African swine fever spread in networks and effectiveness of control strategies,
  \emph{Allen D. Leman Swine Conference-Greeks to Greeks},
  Online.
  \\
\Year{2021}  &
  \Me\
  SPS app, database and modeling – endemic and FADs diseases,
  \emph{National NASAHO ASF Working Group},
  Online.
  \\
\Duration{2021}  &
  \Me\
  SPS app, database and modeling,
  Online.
  https://machado-lab.github.io/rabapp/
  \\
  ~ &
  \emph{National Pork Board},
  \\
  ~ &
  \emph{Illinois Department of Agriculture},
  \\
    ~ &
  \emph{Texas Department of Agriculture},
  \\
      ~ &
  \emph{Oklahoma Department of Agriculture},
  \\
      ~ &
  \emph{Minnesota Department of Agriculture},
  \\
      ~ &
  \emph{North Carolina Department of Agriculture},
  \\
      ~ &
  \emph{Michigan Department of Agriculture},
  \\   
  ~ &
  \emph{Virginia Department of Agriculture},
  \\
  ~ &
    \emph{South Carolina Department of Agriculture},
  \\
  ~ &
    \emph{Nebraska Department of Agriculture},
  \\
  ~ &
    \emph{South Dakota Department of Agriculture},
  \\
  ~ &
    \emph{Wyoming Department of Agriculture},
  \\
  ~ &
      \emph{Arkansas Department of Agriculture},
  \\
  ~ &
      \emph{Pennsylvania Department of Agriculture},
      \\
  ~ &
      \emph{Pennsylvania Farm Bureau and Pennsylvania Pork Producers Council}.
      \\
\Year{2020}  &
  \Me\
  PRRSV outbreaks at the farm-level based on which--biosecurity practices,
  \emph{Zoetis Asia},
  Online.
  \\
  \Year{2020}  &
  \Me\
  On the ability to predict PRRSV outbreaks at the farm-level based on biosecurity practices,
  \emph{Keynote at Allen D. Leman Swine Conference, for the Carlos Pijoan SDEC Symposium: Tightening Biosecurity in Swine Farms },
  Online.
  \\
 \Year{2020}  &
  \Me\
  Modeling PRRSV: Transmission and vaccination strategies,
  \emph{Boehringer Ingelheim},
  Online.
  \\
 \Year{2020}  &
  \Me\
  Introduction to ML with application to disease ecology,
  \emph{Virginia Tech},
  Online.
  \\
\Year{2020}  &
  \Me\
  How technology could help predict future PEDv outbreaks,
  \emph{ANITOX},
  Raleigh, USA.
  \\
\Year{2019}  &
  \Me\
  Disease epidemiology applied to zoonotic diseases,
  \emph{ASTMH American Society of Tropical Medicine  Hygiene},
  Washington DC, USA.
  \\
\Year{2019}  &
  \Me\
  Swine biosecurity,
  \emph{24th  NC Veterinary Conference},
  Raleigh, USA.
  \\
\Year{2018}  &
  \Me\
  Multiscale eco-epidemiological,
  \emph{DARPA PREEMPT},
  Washington DC, USA.
  \\
\Year{2018}  &
  \Me\
  Investigation of Bayesian spatiotemporal models with a case study,
  \emph{EpiQ research group},
  Saint Paul, USA.
  \\
\Year{2018}  &
  \Me\
  Epidemiology for the swine industry,
  \emph{NC Swine Company},
  Raleigh, USA.
  \\
\Year{2018}  &
  \Me\
  Infectious diseases biogeography: the ecological niche modeling approach,
  \emph{Web lecture to Brazilian audience at the Universidade de São Paulo},
  Online.
  \\
\Year{2018}  &
  \Me\
 Spatio-temporal cluster detection for Urban and Rural Leptospirosis in Brazil,
  \emph{Grupo de trabajo en Leptospirosis SVS MS Brasil y PHE OPS},
  Washington DC, USA.
  \\
\Year{2018}  &
  \Me\
  Epidemiological tools and Poultry health,
  \emph{Poultry day},
  Raleigh, USA.
  \\
\Year{2017}  &
  \Me\
  Mapping hotspot of infectious diseases in the animal-human-ecosystem interface,
  \emph{ENDESA Encontro Nacional de Defesa Sanitária Animal},
  Manaus, Brazil.
\end{EntriesTable}


%%%%%%%%%%%%%%%%%%%%%%%%%%%%%%%%%%%%%%%%%%%%%%%%%%%%%%%%%%%%%
\section{Outreach}

\begin{enumerate}
    \item I maintain the Rapid Access Biosecurity (RAB) app standardize Secure Pork Supply (SPS) biosecurity plans, and create maps to visualize the biosecurity infrastructure of individual farms across multiple states
- Current states utilizing \href{https://machado-lab.github.io/rabapp/} {\color{blue}{{}RABapp\textsuperscript{\texttrademark}}}: In total 15 State Animal Health Official (SAHO) are currently using the \href{https://machado-lab.github.io/rabapp/} {\color{blue}{{}RABapp\textsuperscript{\texttrademark}}} or at final arrangements/agreements to utilize the
{\color{blue}{{}RABapp\textsuperscript{\texttrademark}}} . The {\color{blue}{{}RABapp\textsuperscript{\texttrademark}}}  is technology is licensing under NCSU software innovation number (2021-137).

\hspace{0.5 cm }

\item R package:MHASpread: A multi-host Animal Spread Stochastic Multilevel Model(version 0.1.0)
\end{enumerate}

\hspace{0.5 cm }

\textbf{Workshops}

\begin{EntriesTable}
\Year{2023}  &

 \\
\Year{2023}  &
MHASpread to 26 Animal Health Officials of Rio Grande do Sul, Brazil
 \\
\Year{2023}  &
MHASpread to 10 Animal Health Officials of the Sanidad Agropecuaria e Inocuidad Alimentaria (SENASAG), Santa Cruz de la Sierra, Bolivia
 \\
\Year{2023}  &
MHASpread to 17 Animal Health Officials from 7 countries at PANAFTOSA, Rio de Janeiro, Brazil
\Website{https://machado-lab.github.io/PANAFTOSA-Workshop-Rio2023/}
 \\
\Year{2022}  &
MHASpread to 14 Animal Health Officials from 6 countries at PANAFTOSA, Rio de Janeiro, Brazil
\Website{https://machado-lab.github.io/PANAFTOSA-Workshop-Rio2023/}
 \\
\Year{2022}  &
MHASpread to  45 Animal Health Officials of 4 states and PAHO members, Porto Alegre, Brazil
\end{EntriesTable}

%%%%%%%%%%%%%%%%%%%%%%%%%%%%%%%%%%%%%%%%%%%%%%%%%%%%%%%%%%%%%
\section{Grant review activities}

\begin{EntriesTable}
\Year{2020}  &
USDA-NIFA-Critical Agricultural Research and Extension (CARE) (A1701)
 \\
    \Year{2020}  &
The Dutch Research Council (NWO) funds
 \\
 \Year{2019}  &
 Foundation for Food and  Agriculture Research
 \\
  \Year{2019}  &
MRC: Medical Research Council
\end{EntriesTable}

%%%%%%%%%%%%%%%%%%%%%%%%%%%%%%%%%%%%%%%%%%%%%%%%%%%%%%%%%%%%%
\section{Academic Service \& Affiliations}

\begin{EntriesTable}
\Duration{2022}{\Ongoing}  &
Associate editor-Frontiers in Veterinary Science
 \\
\Duration{2020}{\Ongoing}  &
United States Animal Health Association Order Confirmation – USAHA, co-chair of  AAVLD/USAHA Subcommittee on Information Standards
 \\
 \Duration{2020}{\Ongoing}  &
United States Animal Health Association Order Confirmation – USAHA, board member
 \\
 \Duration{2019}{\Ongoing}  &
American Association of Swine Veterinarians (AASV)
 \\
 \Duration{2018}{\Ongoing}  &
 Ecological Society of America
 \\
  \Duration{2018}{\Ongoing}  &
  NC Veterinary Medical Association
\end{EntriesTable}

%%%%%%%%%%%%%
\subsection{Editorial activities(ad-hoc referee)}

\begin{itemize}
  \item Pathology, Research, and Practice
  \item Comparative Clinical Pathology
  \item Microbial Pathogenesis 
  \item Parasitology (Cambridge) 
  \item Scientific Reports 
  \item Frontiers in Microbiology 
  \item Frontiers in Veterinary Science
  \item The American Journal of Tropical Medicine and Hygiene
  \item Research in Veterinary Science
  \item International journal of experimental pathology 
  \item Preventive Veterinary Medicine 
  \item Journal of Wildlife Diseases  
  \item Oxford Research Encyclopedia of Global Public Health
  \item PLOS Neglected Tropical Diseases
  \item Zoonosis 
  \item Nature Communications
  \item Zoonoses and Public Health
  \item Epidemics
  \item Animal
  \item Ciencia Rural
  \item Scientific Data
  \item Journal of the Royal Society Interface
  \item PLOS One
  \item Scientific Data
  \item Journal of Swine Health & Production
  \item PCI Animal Science
  \item Chaos An Interdisciplinary Journal of Nonlinear Science
\end{itemize}

\end{document}
