\documentclass[11pt]{article}

% Full Unicode support for non-ASCII characters
\usepackage[utf8]{inputenc}

% ======================================
% Bibliography
% ======================================
\usepackage[
backend=biber,
maxbibnames=99,
style=numeric,
sorting=ydnt
%citestyle=numeric,
]{biblatex}

%% this will count the references and reverse the order
\addbibresource{citations.bib} %Import the bibliography file
% Count total number of entries in each refsection
\AtDataInput{%
  \csnumgdef{entrycount:\therefsection}{%
    \csuse{entrycount:\therefsection}+1}}

% Print the label number as the total number of entries in the current ref section, minus the actual labelnumber, plus one
\DeclareFieldFormat{labelnumber}{\mkbibdesc{#1}}    
\newrobustcmd*{\mkbibdesc}[1]{%
  \number\numexpr\csuse{entrycount:\therefsection}+1-#1}

%%%%%%%%%%%%%%%%%%%%%%%%%%%%%%%%%%%%%%%%%%%%%%%%%%%
%% Your information %%%%%%%%%%%%%%%%%%%%%%%%%%%%%%%
%%%%%%%%%%%%%%%%%%%%%%%%%%%%%%%%%%%%%%%%%%%%%%%%%%%

% Useful aliases for main subsessions
\newcommand{\NCSU}{North Carolina State University}
\newcommand{\vet}{College of Veterinary Medicine}
\newcommand{\UHEARTH}{Department of Earth Sciences}
\newcommand{\CVM}{Department of Population Health and Pathobiology}

% Identifying information
\newcommand{\Title}{Curriculum Vit\ae}
\newcommand{\FirstName}{Gustavo}
\newcommand{\LastName}{Machado}
\newcommand{\Initials}{}
\newcommand{\MyName}{\FirstName\ \LastName}
\newcommand{\Me}{\textbf{\LastName, \FirstName \Initials }} 
% For citations
\newcommand{\Email}{gmachad@ncsu.edu}
\newcommand{\PersonalWebsite}{machado-lab.github.io}
\newcommand{\LabWebsite}{machado-lab.github.io}
\newcommand{\ORCID}{0000-0001-7552-6144}
\newcommand{\Affiliation}{\CVM \\ \vet \\ \NCSU}
\newcommand{\Address}{
  1060 William Moore Drive, Raleigh NC 27607, USA
}

% Packages

% Metadata for the PDF output and control of hyperlinks
\usepackage[colorlinks=true]{hyperref}
\hypersetup{
  pdftitle={\MyName\ - \Title},
  pdfauthor={\MyName},
  linkcolor=blue,
  citecolor=blue,
  filecolor=black,
 urlcolor=MidnightBlue
}

\usepackage{url} % Evitar enlaces en bibliografía largos & enlazar enlaces de bibliografía
\usepackage{breakurl} % Broken links
\usepackage{etoolbox}

% Template configuration
%%%%%%%%%%%%%%%%%%%%%%%%%%%%%%%%%%%%%%%%%%%%%%%%%%%%%%%%

% Disable hyphenation
\usepackage[none]{hyphenat}

% Control the font size
\usepackage{anyfontsize}

% Icon fonts (requires using xelatex or luatex)
\usepackage[fixed]{fontawesome5}
\usepackage{academicons}

% Template variables for styling
\newcommand{\TablePad}{\vspace{-0.4cm}}
\newcommand{\SoftwareTitle}[1]{{\bfseries #1}}
\newcommand{\TableTitle}[1]{{\fontsize{12pt}{0}\selectfont \itshape #1}}

% For fancy and multi page tables
\usepackage{tabularx}
\usepackage{ltablex}

% Define a new environment to place all CV entries in a 2-column table.
% Left column are the dates, right column the entries.
\usepackage{environ}
\NewEnviron{EntriesTable}{
\TablePad
\begin{tabularx}{\textwidth}{@{}p{0.12\textwidth}@{\hspace{0.02\textwidth}}p{0.86\textwidth}@{}}
  \BODY
\end{tabularx}
}

% Macros to add links and mark publications
\newcommand{\DOI}[1]{doi:\href{https://doi.org/#1}{#1}}
\newcommand{\DOILink}[1]{\href{https://doi.org/#1}{doi.org/#1}}
\newcommand{\Preprint}[1]{\newline • Preprint: \faFilePdf\ \DOILink{#1}}
\newcommand{\Youtube}[1]{\newline • Recording: \faYoutube\, \href{https://www.youtube.com/watch?v=#1}{youtube.com/watch?v=#1}}
\newcommand{\GitHub}[1]{\newline • Code: \faGithub\ \href{https://github.com/#1}{#1}}
\newcommand{\Role}[1]{\newline • Role: #1}
\newcommand{\Website}[1]{\newline • Website: \href{https://#1}{#1}}
\newcommand{\Slides}[1]{\newline • Slides: \faTv\ \href{https://#1}{#1}}
\newcommand{\SlidesDOI}[1]{\newline • Slides: \faTv\ \DOILink{#1}}
\newcommand{\PosterDOI}[1]{\newline • Poster: \faImage\ \DOILink{#1}}
\newcommand{\OA}{\aiOpenAccess\enspace}
\newcommand{\Invited}{\newline • \textbf{Invited talk}}

% Macros to set the year and duration on the left column
\newcommand{\Duration}[2]{\fontsize{10pt}{0}\selectfont #1 -- #2}
\newcommand{\Year}[1]{\fontsize{10pt}{0}\selectfont #1}
\newcommand{\Ongoing}{present}
%\newcommand{\Ongoing}{$\ast$}
\newcommand{\Future}{future}
\newcommand{\Review}{in review}
\newcommand{\Accepted}{accepted}
\newcommand{\Appointment}[4]{\textbf{#1} \newline #2 \newline #3 \newline #4}

% Define command to insert month name and year as date
\usepackage{datetime}
\newdateformat{monthyear}{\monthname[\THEMONTH], \THEYEAR}

% Set the page margins
\usepackage[a4paper,margin=1.5cm,includehead,headsep=5mm]{geometry}

% To get the total page numbers (\pageref{LastPage})
\usepackage{lastpage}

% No indentation
\setlength\parindent{0cm}

% Increase the line spacing
\renewcommand{\baselinestretch}{1.1}
% and the spacing between rows in tables
\renewcommand{\arraystretch}{1.5}

% Remove space between items in itemize and enumerate
\usepackage{enumitem}
\setlist{nosep}

% Use custom colors
\usepackage[usenames,dvipsnames]{xcolor}

% Set fonts. Requires compilation with xelatex
\usepackage{fontspec}  % required to make older xelatex compile with UTF8

% Configure the font style for sections
\usepackage{sectsty}
\sectionfont{\vspace{0.5cm}\bfseries\fontsize{12pt}{0}\selectfont\uppercase}
\subsectionfont{\vspace{0.2cm}\mdseries\fontsize{12pt}{0}\selectfont\uppercase}

% Set the spacing for sections
%\usepackage{titlesec}
%\titlespacing{\section}{0pt}{0cm}{0.3cm}
%\titlespacing{\subsection}{0pt}{0.3cm}{0.3cm}

% Disable number of sections. Use this instead of "section*" so that the sections still
% appear as PDF bookmarks. Otherwise, would have to add the table of contents entries
% manually.
\makeatletter
\renewcommand{\@seccntformat}[1]{}
\makeatother

% Set fancy headers
\usepackage{fancyhdr}
\pagestyle{fancy}
\fancyhf{}
\chead{
  \fontsize{10pt}{12pt}\selectfont
  \MyName
  \hspace{0.2cm} -- \hspace{0.2cm}
  \Title
  \hspace{0.2cm} -- \hspace{0.2cm}
  \monthyear\today
}
\rhead{\fontsize{10pt}{0}\selectfont \thepage/\pageref*{LastPage}}
\renewcommand{\headrulewidth}{0pt}


%%%%%%%%%%%%%%%%%%%%%%%%%%%%%%%%%%%%%%%%%%%%%%%%%%%%%%%%%%%%%%%%%%%%%%%%%%%%%%%


\begin{document}

% No header for the first page
\thispagestyle{empty}

%%%%%%%%%%%%%%%%%%%%%%%%%%%%%%%%%%%%%%%%%%%%%%%%%%%%%%%%%%%%%%%%%%%%%%%%%%%%%%%
% HEADER
{\fontsize{22pt}{0}\selectfont\MyName}\\[-0.1cm]
\rule{\textwidth}{0.2pt}
\begin{minipage}[t]{0.595\textwidth}
  \Affiliation
  \\
  \Address
\end{minipage}
\begin{minipage}[t]{0.405\textwidth}
  \begin{flushright}
  Last updated: \monthyear\today
  \\
    ORCID: \href{https://orcid.org/\ORCID}{\ORCID}
    \\
    email: \href{mailto:\Email}{\Email}
    \\
    Lab web site: \href{https://www.\LabWebsite}{\LabWebsite}
  \end{flushright}
\end{minipage}

%%%%%%%%%%%%%%%%%%%%%%%%%%%%%%%%%%%%%%%%%%%%%%%%%%%%%%%%%%%%%%%%%%%%%%%%%%%%%%%
\section{Professional Appointments}

\begin{EntriesTable}
  \Duration{2018}{\Ongoing}  &
  \Appointment{Assistant Professor}{\CVM}{\vet}{\NCSU, USA}
  \\
  \Duration{2016}{2017}  &
  \Appointment{Postdoctoral researcher}{Department of Veterinary Population Medicine }{College of Veterinary Medicine }{University of Minnesota, USA}
  \\
  \Year{2016}  &
  \Appointment{Assistant Professor}{Departamento de Medicina Veterinaria Preventiva}{Faculdade de Medicina Veterinaria}{Universidade Federal do Rio Grande do Sul, Brazil}
\end{EntriesTable}


%%%%%%%%%%%%%%%%%%%%%%%%%%%%%%%%%%%%%%%%%%%%%%%%%%%%%%%%%%%%%%%%%%%%%%%%%%%%%%%
\section{Education}

\begin{EntriesTable}
  \Duration{2013}{2016}  &
  \textbf{PhD in Veterinary Epidemiology}, Universidade Federal do Rio Grande do Sul, Brazil.
  \\
  \Duration{2011}{2013}  &
  \textbf{MVSc in Veterinary Epidemiology}, Universidade Federal do Rio Grande do Sul, Brazil.
  \\
  \Duration{2003}{2010}  &
  \textbf{DVM Veterinary Medicine}, Universidade Federal de Santa Maria, Brazil.
\end{EntriesTable}

%%%%%%%%%%%%%%%%%%%%%%%%%%%%%%%%%%%%%%%%%%%%%%%%%%%%%%%%%%%%%%%%%%%%%%%%%%%%%%%
\section{Funding}
Ongoing research support
\begin{EntriesTable}
  \Duration{2022}{2024}  &
  \textbf{PI}: \Me\
  ``Advanced interpretable machine learning for on-farm swine biosecurity with social network models on pig movement data'' \textbf {Funded by:} Foundation for Food and Agriculture Research (FFAR)-New Innovator Fellowship``
  \textit{Amount \$450,000}. Award ID: {TBD}.
  \\
  \Duration{2021}{2022}  &
  \textbf{PI}: \Me\
  ``Combining standardized on-farm biosecurity plans with animal movement data in a user-friendly rapid access biosecurity management tool: a multi state study'' \textbf {Funded by:}
  Animal and Plant Health Inspection Service (USDA-APHIS)-National Animal Disease Preparedness and Response Program (NADPRP)``
  \textit{Amount \$182,301}. Award ID: {APP-15216}.
  \\
  \Duration{2020}{2024}  &
  \textbf{PI}: \Me\
  ``Near-real time spatiotemporal resource allocation to improve swine health'' \textbf {Funded by:} USDA-AFRI Foundational and Applied Science``
  \textit{Amount \$500,000}. Award ID: {20206702132462}.
  \\
  \Duration{2020}{2021}  &
  \textbf{PI}: \Me\
  ``Dynamic transmission modeling-the role of feed, feed ingredients in swine disease transmission'' \textbf {Funded by:} Fats and Proteins Research Foundation``
  \textit{Amount \$76,000}. Award ID: {2021-0715}.
  \\
  \Duration{2021}{2022}  &
  \textbf{PI}: \Me\
  ``Dynamic transmission modeling-the role of feed, feed ingredients in swine disease transmission'' \textbf {Funded by:} FUNDESA-Fundo de Desenvolvimento e Defesa Animal``
  \textit{Amount \$114,000}. Award ID: {2021-0715}.
  \\
  \Year{2021}  &
  \textbf{PI}: \Me\
  ``Foot and mouth disease in Brazil-cattle, swine, small ruminants and poultry (Aya Omar-DVM Student)'' \textbf {Funded by:} Triangle Community Foundation``
  \textit{Amount \$5,000}. Award ID: {3425}.
  \\
  \Duration{2019}{2022}  &
  \textbf{PI}: \Me\
  ``Assessing biosecurity vulnerabilities to predict the risk of new Porcine Reproductive and Respiratory Syndrome outbreaks'' \textbf {Funded by:} USDA-AFRI Foundational and Applied Science``
  \textit{Amount \$300,000}. Award ID: {20196800829910}.
\end{EntriesTable}

Completed research support
\begin{EntriesTable}
  \Duration{2019}{2020} &
  \textbf{PI}: \Me\
  ``High-resolution dynamic risk mapping to guide timely disease interventions- Swine health information center'' \textbf {Funded by:} SHIC-Swine Health Information 
  \textit{Amount \$89,987}. Award ID: {17-141}.
  \\
  \Duration{2019}{2020}  &
  \textbf{PI}: \Me\
  ``Network analysis to guide active surveillance'' \textbf {Funded by:} FUNDESA-Fundo de Desenvolvimento e Defesa Animal``
  \textit{Amount \$22,696}. Award ID: {2019-049}.
 \\
  \Duration{2019}{2020}  &
  {PI}: Linhares, \textbf{Co-PI} \Me\
  ``Biosecurity screening tool to identify breeding herds’ risk to PRRS outbreak using a short survey Swine health information center'' \textbf {Funded by:} SHIC-Swine Health Information Center``
  \textit{Amount \$75,201}. Award ID: {2019-2601}.
  \\
  \Duration{2019}{2020}  &
  {PI}: Goss, \textbf{Co-PI} \Me\
  ``Distribution and activity of Pythium insidiosum in the Chincoteague National Wildlife Refuge'' \textbf {Funded by:} U.S. Fish and Wildlife Service``
  \textit{Amount \$11,292}. Award ID: {2019-2012}.
  \\
 \Duration{2018}{2020}  &
  \textbf{PI}: VanderWall, \textbf{Co-PI} \Me\
  ``Forecasting the spread of endemic viruses of swine in the United States'' \textbf {Funded by:} USDA-AFRI Foundational and Applied Science``
  \textit{Amount \$300,000}. Award ID: {XX}.
  \\
  \Duration{2021}{2022}  &
  \textbf{PI}: \Me\
  ``Development of a machine-learning framework to identify risk factors for COVID-19 infection in swine caretakers and estimate the chances of new COVID-19 waves`` \textbf {Funded by:} Grant Competition (UGPN)``
  \textit{Amount \$30,000}. Award ID: {14}.
  \\
    \Duration{2019}{2020}  &
  \textbf{PI}: \Me\
  ``Development of a dissemination platform for temporal, spatial and phylogenetic analysis of Avian Infectious Bronchitis Virus sequences`` \textbf {Funded by:} NCSU Research and Innovation Seed Funding Program``
  \textit{Amount \$18,750}. Award ID: {2019-2366}.
  \\
 \Duration{2019}{2020}  &
  \textbf{PI}: \Me\
  ``Identification of key factors enabling PRRSv spatial distribution and the importance of new virus variant introduction on the incidence of PRRSv in North Carolina`` \textbf {Funded by:} NCSU College of Veterinary Medicine``
  \textit{Amount \$25,000}. Award ID: {14}.
  \\
   \Duration{2019}{2020}  &
  \textbf{PI}: \Me\
  ``Use of swine movement information to improve risk-based surveillance`` \textbf {Funded by:} CVM- Global Health grant program``
  \textit{Amount \$19,792}. Award ID: {14}.
  \\
  \Duration{2021}{2022}  &
  \textbf{PI}: \Me\
  ``Optimizing control interventions for Visceral Leishmaniasis in multiple settings`` \textbf {Funded by:} Grant Competition (UGPN)``
  \textit{Amount \$30,000}. Award ID: {14}.
\end{EntriesTable}

%%%%%%%%%%%%%%%%%%%%%%%%%%%%%%%%%%%%%%%%%%%%%%%%%%%%%%%%%%%%%%%%%%%%%%%%%%%%%%%
\section{Awards \& Honors}

\begin{EntriesTable}
   \Duration{2021}{2024}  &
  Foundation for Food and Agriculture Research (FFAR)-New Innovator Fellowship, \textit{Amount \$450,000.}
  \\
 \Duration{2013}{2016}  &
  Brazilian Ministry of Education CAPES \textbf{Masters Research Scholarship, 3 years funding.}
  \\
  \Duration{2011}{2013}  &
  Brazilian Ministry of Education CAPES \textbf{Masters Research Scholarship, 3 years funding.}
  \\
  \Duration{2006}{2007}  &
 MAST International | MAST International at University of Minnesota (US) \textbf{International exchange student.}
  \\
  \Duration{2009}{2011}  &
  Brazilian Ministry of Education CAPES \textbf{DVM Research
  Scholarship.}
\end{EntriesTable}


%%%%%%%%%%%%%%%%%%%%%%%%%%%%%%%%%%%%%%%%%%%%%%%%%%%%%%%%%%%%%%%%%%%%%%%%%%%%%%%
% ======================================
% PUBLICATIONS
% ======================================

\section{Publications}

\subsection{Preprint}
\begin{EntriesTable}
\Year{2021}  &
  Cardenas, Nicolas C, Sykes, Abagael L, Lopes, Francisco PN, \Me.\
  Multiplespecies animal movements: network properties, disease dynamic and the impact of targeted controlactions.
  \emph{arXiv}.
  \Preprint{arXiv:2107.10108 }
  %\GitHub{}
  \\
\Year{2021}  &
  Sykes, Abagael L, Silva, Gustavo S, Holtkamp, Derald J, Mauch, Broc W, Osemeke, Onyekachukwu, Linhares, Daniel CL, \Me.\
  Interpretable machine learning applied to on-farm biosecurity and porcine reproductive and respiratory syndrome virus.
  \emph{arXiv}.
  \Preprint{arXiv:2107.10108(2021)}
  \GitHub{https://nfj1380.github.io/mrIML/}
  \\
\Year{2021}  &
  Galvis, Jason A, Corzo, Cesar, \Me.\
  Modelling porcine reproductive and respiratory syndrome virus dynamics to quantify the contribution of multiple modes of transmission: between-farm animal and vehicle movements, farm-to-farm proximity, feed ingredients, and re-break.
  \emph{bioRxiv}.
  \Preprint{10.1101/2021.07.26.453902)}
  %\GitHub{}
  \\
\Year{2021}  &
  \Me\, Farthing, Trevor, Andraud, Mathieu, Lopes, Lopes, Francisco PN, Lanzas, Cristina.
  Modeling the role of mortality-based response triggers on the effectiveness of {African} swine fever control strategies.
  \emph{bioRxiv}.
  \Preprint{10.1101/2021.04.05.438400}
  \GitHub{https://github.com/machado-lab/ASF_model}
  
\end{EntriesTable}

%% cite all other papers
\nocite{*}
\printbibliography[title=PEER-REVIEWED]

\section{Open-source Software}

\begin{EntriesTable}
  \Duration{2021}{\Ongoing} &
  \textbf{MrIML}
  \newline
  Multivariate (multi-response) interpretable machine learning
  \Role{Creator and core developer}
  \GitHub{nfj1380/mrIML/}
  \Website{nfj1380.github.io/mrIML/index.html}
  \\
  \Duration{2020}{\Ongoing} &
  \textbf{Research laboratory Jekyll site (Ruby)}
  \newline
  Research laboratory
  \Role{Creator, main developer, project leadership}
  \Website{machado-lab.github.io}
\end{EntriesTable}

%%%%%%%%%%%%%%%%%%%%%%%%%%%%%%%%%%%%%%%%%%%%%%%%%%%%%%%%%%%%%%%%%%%%%%%%%%%%%%%
\section{Teaching}

\subsection{DVM-level}

\begin{EntriesTable}
  \Duration{2021}{\Ongoing}  &
  VMP 971 Food Animal Diagnostics for Disease Diagnosis, Control, and Population Surveillance.
  \textit{\LIV}
  \\
  \Duration{2020}{\Ongoing}  &
  VMP 979 Epidemiology, Public Health, and Public Policy.
  \textit{\LIV}
  \\
  \Duration{2020}{\Ongoing}  &
  VMP-991-151 Trade and Globalization (Guest lecturer).
  \textit{\LIV}
  \\
  \Duration{2018}{\Ongoing}  &
  VMP 991 Special Topics in PHP, Transboundary Disease and Spatial Epidemiology.
  \textit{\LIV}
  \\
  \Duration{2019}{\Ongoing}  &
  VMP 904 Swine Industry (Guest lecturer).
  \textit{\LIV}
  \\
 \Year{2018}  &
  VMP 945 Epidemiology and Public Health.
\end{EntriesTable}

\subsection{Graduate-level}

\begin{EntriesTable}
  \Duration{2021}{\Ongoing}  &
  CBS 775 (001) Spring 2021 Designing population-based research.
  \textit{\LIV}
  \\
  \Duration{2019}{\Ongoing}  &
  CBS 650/565 Fundamentals of biomedical sciences.
  \textit{\LIV}
  \\
  \Duration{2018}{\Ongoing}  &
  CBS 595 Special Topics: Epidemiology I.
  \textit{\LIV}
  \\
  \Duration{2019}{\Ongoing}  &
  CBS 595 Independent studies in epidemiology.
  \textit{\LIV}
  \\
  \Duration{2019}{\Ongoing}  &
  VMP 904 Swine Industry (Guest lecturer).
  \textit{\LIV}
\end{EntriesTable}

\subsection{Invited presentations}

\begin{EntriesTable}
%\Year{future}  &
\Year{2021} &
  The Generic Mapping Tools for Geodesy
  \newline
  \textit{UNAVCO} (online)
  \GitHub{GenericMappingTools/2021-unavco-course}
  \\
\end{EntriesTable}


%%%%%%%%%%%%%%%%%%%%%%%%%%%%%%%%%%%%%%%%%%%%%%%%%%%%%%%%%%%%%%%%%%%%%%%%%%%%%%%
\section{Supervision and mentoring}

\subsection{Postdoctoral researchers}

\begin{EntriesTable}
\Duration{2021}{\Ongoing}  &
  Dr. Nicolas Cardenas
  \newline
  NC State University
\\
\Duration{2020}{\Ongoing}  &
  Dr. Jason Galvis
  \newline
  NC State University
\\
\Duration{2018}{2020}  &
  Dr. Manuel Jara
  \newline
  NC State University
\end{EntriesTable}

\subsection{P\lowercase{h}D}

\begin{EntriesTable}
\Duration{2021}{\Ongoing}  &
  Abagael Sykes (Chair)
  \newline
  NC State University-CBS
\\
\Duration{2021}{\Ongoing}  &
  Felipe Sanchez (Chair)
  \newline
  NC State University-CGA
\\
\Duration{2021}{\Ongoing}  &
  Parker Trostle (co-chair)
  \newline
  NC State University-STA
\end{EntriesTable}

\subsection{Master's}

\begin{EntriesTable}
\Duration{2021}{\Ongoing}  &
  Cameron Ellington (co-chair)
  \newline
  NC State University-resident
\\
\Duration{2020} &
  Kelsey Mills (Chair)
  \newline
  NC State University-CGA
\\
  \Duration{2020} &
 Heather Paxson  (advisor)
  \newline
  NC State University-CGA
\\
\Duration{2020} &
 Patrícia Warzensaky Gottardo Balestrin (Co-chair)
  \newline
  UDESC-Brazil
\end{EntriesTable}

\subsection{DVM-students}

\begin{EntriesTable}
\Duration{2021}{\Ongoing}  &
  Alyssa Valentine
  \newline
  \\
\Duration{2021}{\Ongoing}  &
  Elizabeth Farren Walsh
  \newline
  \\
\Year{2020}  &
  Aya Omar
  \newline
  \\
\Duration{2018}{2021}  &
  Stephanie Krasteva
  \newline
  \\
\Duration{2019}{2020}  &
  Jamie Madigan
  \newline
  \\
\Duration{2019}{2020}  &
  Jack Lee Miller
  \newline
\end{EntriesTable}


\subsection{Undergraduate}

\begin{EntriesTable}
\Duration{2020}{\Ongoing}  &
  Allyson Freeman
  \newline
  \\
\Duration{2021}{\Ongoing}  &
  Madison Joyce
  \newline
  \\
\Year{2021}  &
  Alyssa White
  \newline
  \\
\Year{2021}  &
  Grace Winesett
  \newline
  \\
\Duration{2019}{2020}  &
  Victoria J. Reynolds
  \newline
  \\
\Year{2020}  &
  Brittany Lee
  \newline
  \\
\Year{2018} &
  Bridget Knapp
  \newline
\end{EntriesTable}

%%%%%%%%%%%%%%%%%%%%%%%%%%%%%%%%%%%%%%%%%%%%%%%%%%%%%%%%%%%%%%%%%%%%%%%%%%%%%%%
\section{Presentations}

\begin{EntriesTable}
\Year{2021}  &
  \Me.
  Academia e software livre: Desafios e oportunidades no Brasil e no exterior,
  \emph{National Observatory's SEG and EAGE Student Chapter},
  Rio de Janeiro, Brazil.
  \Invited{}
  \GitHub{leouieda/2021-07-22-on}
  \Youtube{r2x-DN6laj8}
  \\
  ~ &
  \Me, \Santiago, \Agustina.
  Open-science for gravimetry: tools, challenges, and opportunities,
  \emph{GFZ Helmholtz Centre Potsdam},
  Germany.
  \Invited{}
  \GitHub{leouieda/2021-06-22-gfz}
  \SlidesDOI{10.6084/m9.figshare.14838477}
  \Youtube{z-5dvWfB\_SM}
  \\
  ~ &
  \Me, \Santiago, \Agustina.
  Fatiando a Terra: Open-source tools for geophysics,
  \emph{Geophysical Society of Houston},
  Houston, USA.
  \Invited{}
  \GitHub{fatiando/2021-gsh}
  \\
  ~ &
  \Me, \Santiago, \Agustina, \LPerozzi, \MWieczorek.
  Harmonica and Boule: Modern Python tools for geophysical gravimetry,
  \emph{EGU 2021},
  Online.
  \DOI{10.5194/egusphere-egu21-8291}.
  \GitHub{fatiando/egu2021}
  \\
\Year{2020}  &
  \Me.
  Geophysical research powered by open-source,
  \emph{Christian Albrechts Universität zu Kiel},
  Kiel, Germany.
  \Invited
  \Slides{www.leouieda.com/2020-07-01-kiel}
  \\
  ~ &
  \Me.
  Geophysical research powered by open-source,
  \emph{Departamento de Geofísica, IAG, Universidade de São Paulo},
  São Paulo, Brazil.
  \Invited
  \Youtube{VqI8BX1Yg54}
  \Slides{www.leouieda.com/2020-06-18-usp}
  \\
  ~ &
  \Me.
  Geophysical research powered by open-source,
  \emph{Technische Universität Bergakademie Freiberg},
  Freiberg, Germany.
  \Invited
  \Slides{www.leouieda.com/2020-06-04-freiberg}
  \\
  ~ &
  \Me.
  Geophysical research powered by open-source,
  \emph{Geographic Data Science Lab, University of Liverpool},
  Liverpool, UK.
  \Invited
  \Slides{www.leouieda.com/liverpool-gdsl-2020}
  \\
  ~ &
  \Me, \Santiago.
  Evaluating the accuracy of equivalent-source predictions using
  cross-validation,
  \emph{EGU 2020},
  Vienna, Austria.
  \DOI{10.5194/egusphere-egu2020-15729}.
  \SlidesDOI{10.6084/m9.figshare.12245372}
  \\
\Year{2019}  &
  \Me, \Paul.
  PyGMT: Accessing the Generic Mapping Tools from Python,
  \emph{AGU 2019},
  San Francisco, USA.
  \PosterDOI{10.6084/m9.figshare.11320280}
  \\
  ~ &
  \Me.
  Building the foundations for open-source geophysics,
  \emph{\CVM, \LIV},
  UK.
  \SlidesDOI{10.6084/m9.figshare.10255832}
  \\
\Year{2018}  &
  \Me, \Eric, \Paul, \David.
  Coupled Interpolation of Three-component GPS Velocities,
  \emph{AGU 2018},
  Washington DC, USA.
  \PosterDOI{10.6084/m9.figshare.7440683}
  \\
  ~ &
  \Me.
  Machine Learning Lessons for Geophysics,
  \emph{Department of Earth Sciences, \UHM},
  Honolulu, USA.
  \SlidesDOI{10.6084/m9.figshare.7203344}
  \\
  ~ &
  \Me, \Paul.
  Building an object-oriented Python interface for the Generic Mapping Tools,
  \emph{Scipy 2018},
  Austin, USA.
  \Youtube{6wMtfZXfTRM}
  \SlidesDOI{10.6084/m9.figshare.6814052}
  \\
  ~ &
  \Me, \David, \Paul.
  Joint Interpolation of 3-component GPS Velocities Constrained by
  Elasticity,
  \emph{AOGS $15^{th}$ Annual Meeting},
  Honolulu, USA.
  \SlidesDOI{10.6084/m9.figshare.6387467}
  \\
  ~ &
  \Me, \Paul.
  Integrating the Generic Mapping Tools with the Scientific Python Ecosystem,
  \emph{AOGS $15^{th}$ Annual Meeting},
  Honolulu, USA.
  \PosterDOI{10.6084/m9.figshare.6399944}
  \\
\Year{2017}  &
  \Me, \Paul.
  Nurturing reliable and robust open-source scientific software,
  \emph{AGU Fall Meeting 2017},
  New Orleans, USA.
  \Invited
  \Youtube{0GO4ZZ5Ry6M}
  \\
  ~  &
  \Me, \Paul.
  A modern Python interface for the Generic Mapping Tools,
  \emph{AGU Fall Meeting 2017},
  New Orleans, USA.
  \PosterDOI{10.6084/m9.figshare.5662411}
  \\
  ~  &
  \Me, \Paul.
  Bringing the Generic Mapping Tools to Python,
  \emph{Scipy 2017},
  Austin, USA.
  \Youtube{93M4How7R24}
  \SlidesDOI{10.6084/m9.figshare.7635833}
  \\
  ~ &
  \Me.
  Inverting gravity to map the Moho: A new method and the open source
  software that made it possible,
  \emph{Department of Geology and Geophysics, \UHM},
  Honolulu, USA.
  \SlidesDOI{10.6084/m9.figshare.4779766}
  \\
\Year{2016}  &
  \Me.
  Fatiando a Terra: construindo uma base para ensino e pesquisa de geofísica,
  \emph{Observatório Nacional},
  Rio de Janeiro, Brazil.
  \Invited
  \SlidesDOI{10.6084/m9.figshare.1381870}
  \\
\Year{2015}  &
  \Me.
  Fatiando a Terra: construindo uma base para ensino e pesquisa de geofísica,
  \emph{Universidade de São Paulo},
  São Paulo, Brazil.
  \Invited
  \SlidesDOI{10.6084/m9.figshare.1381870}
  \\
\Year{2014}  &
  \Me, \Bi, \Val.
  Using Fatiando a Terra to solve inverse problems in geophysics,
  \emph{Scipy 2014},
  Austin, USA.
  \PosterDOI{10.6084/m9.figshare.1089987}
  \\
  ~ &
  \Me, \Val.
  Gravity inversion in spherical coordinates using tesseroids,
  \emph{EGU General Assembly 2014},
  Vienna, Austria.
  \SlidesDOI{10.6084/m9.figshare.1155457}
  \\
\Year{2013}  &
  \Me, \Bi, \Val.
  Modeling the Earth with Fatiando a Terra,
  \emph{Scipy 2013},
  Austin, USA.
  \DOI{10.25080/Majora-8b375195-010}.
  \Youtube{Ec38h1oB8cc}
  \Slides{www.leouieda.com/scipy2013/?theme=night}
  \\
  ~ &
  \Me, \Val.
  3D magnetic inversion by planting anomalous densities,
  \emph{AGU Meeting of the Americas},
  Cancun, Mexico.
  \SlidesDOI{10.6084/m9.figshare.703651}
  \\
\Year{2012}  &
  \Dio, \Me, \YLi, \Val, \BragaVale, \Angeli, \Peres.
  Iron ore interpretation using gravity-gradient inversions in the Carajás,
  Brazil,
  \emph{SEG Annual Meeting 2012},
  Las Vegas, USA.
  \DOI{10.1190/segam2012-0525.1}.
  \SlidesDOI{10.6084/m9.figshare.156865}
  \\
  ~ &
  \Me, \Val.
  Use of the ``shape-of-anomaly'' data misfit in 3D inversion by planting
  anomalous densities,
  \emph{SEG Annual Meeting 2012},
  Las Vegas, USA.
  \DOI{10.1190/segam2012-0383.1}.
  \SlidesDOI{10.6084/m9.figshare.156864}
  \\
  ~ &
  \Me, \Val.
  Rapid 3D inversion of gravity and gravity gradient data to test geologic
  hypotheses,
  \emph{International Symposium on Gravity, Geoid and Height Systems},
  Venice, Italy.
  \SlidesDOI{10.6084/m9.figshare.156859}
  \\
\Year{2011}  &
  \Me, \Val.
  Robust 3D gravity gradient inversion by planting anomalous densities,
  \emph{SEG Annual Meeting 2011},
  San Antonio, USA.
  \DOI{10.1190/1.3628201}.
  \SlidesDOI{10.6084/m9.figshare.156863}
  \\
  ~ &
  \Me, \Val.
  3D gravity inversion by planting anomalous densities,
  \emph{Internation Congress of the Brazilian Geophysical Society},
  Rio de Janeiro, Brazil.
  \DOI{10.1190/sbgf2011-179}.
  \SlidesDOI{10.6084/m9.figshare.156861}
  \\
  ~ &
  \Me, \Everton, \Carla, \Eder.
  Optimal forward calculation method of the Marussi tensor due to a geologic
  structure at GOCE height,
  \emph{4th International GOCE User Workshop},
  Munich, Germany.
  \PosterDOI{10.6084/m9.figshare.92624}
  \\
  ~ &
  \Me, \Val.
  3D gravity gradient inversion by planting density anomalies,
  \emph{73th EAGE Conference and Exhibition incorporating SPE EUROPEC},
  Vienna, Austria.
  \DOI{10.3997/2214-4609.20149567}.
  \PosterDOI{10.6084/m9.figshare.91511}
  \\
\Year{2010}  &
  \Me, \Naomi, \Carla.
  Computation of the gravity gradient tensor due to topographic masses using
  tesseroids,
  \emph{AGU Meeting of the Americas},
  Foz do Iguaçu, Brazil.
  \SlidesDOI{10.6084/m9.figshare.156858}
  \\
\Year{2008}  &
  \Me, \Naomi.
  Utilização de tesseróides na modelagem de dados de gradiometria
  gravimétrica,
  \emph{XIII Simpósio de Iniciação Científica do IAG-USP},
  São Paulo, Brazil.
  \PosterDOI{10.6084/m9.figshare.4779760}
  \\
\Year{2006}  &
  \Me, \Manoel.
  Paleomagnetismo e mineralogia magnética dos diques cambrianos de Maravilhas
  e Prata (PB),
  \emph{XI Simpósio de Iniciação Científica do IAG/USP},
  São Paulo, Brazil.
  \PosterDOI{10.6084/m9.figshare.4779769}
\end{EntriesTable}


%%%%%%%%%%%%%%%%%%%%%%%%%%%%%%%%%%%%%%%%%%%%%%%%%%%%%%%%%%%%%%%%%%%%%%%%%%%%%%%
\section{Outreach}

I maintain a blog about my research, geoscience, and programming at
\href{https://www.leouieda.com/blog}{leouieda.com/blog}
\\

\begin{EntriesTable}
\Year{2018}  &
  Interviewed by the geoscience podcast \textit{Don't Panic Geocast}, episode 166
  \textit{``You are headed to a warm and sunny place''}:
  \href{http://www.dontpanicgeocast.com/?p=638}{dontpanicgeocast.com/?p=638}
  \\
\Year{2017}  &
  Volunteer for the \textit{Hour of Code} at Salt Lake Elementary School, Honolulu,
  USA.
  \\
\Year{2016}  &
  Interviewed by the geoscience podcast \textit{Undersampled Radio}, episode
  \textit{``Open Sourcery''}:
  \href{https://undersampledrad.io/home/2016/7/open-sourcery}{undersampledrad.io/home/2016/7/open-sourcery}
\end{EntriesTable}

Geophysical tutorials for the SEG publication \textit{The Leading Edge}:
\\

\begin{EntriesTable}
\Year{2017}  &
  \OA
  \Me.
  Step-by-step NMO correction,
  \emph{The Leading Edge},
  \DOI{10.1190/tle36020179.1}.
  \GitHub{pinga-lab/nmo-tutorial}
  \\
\Year{2014}  &
  \OA
  \Me, \Bi, \Val.
  Geophysical tutorial: Euler deconvolution of potential-field data,
  \emph{The Leading Edge},
  \DOI{10.1190/tle33040448.1}.
  \GitHub{pinga-lab/paper-tle-euler-tutorial}
\end{EntriesTable}


%%%%%%%%%%%%%%%%%%%%%%%%%%%%%%%%%%%%%%%%%%%%%%%%%%%%%%%%%%%%%%%%%%%%%%%%%%%%%%%
\section{Academic Service \& Affiliations}

\subsection{Editor}

\begin{EntriesTable}
  \Duration{2019}{\Ongoing} & Topic editor for the \textit{Journal of Open Source Software}
\end{EntriesTable}

\subsection{Community service}

\begin{EntriesTable}
  \Duration{2019}{\Ongoing} & EarthArXiv Advisory Council
\end{EntriesTable}

\subsection{Committees}

\begin{EntriesTable}
\Duration{2020}{\Ongoing} &
  Department committee for web presence (website, social media, etc),
  \LIV.
  \\
\Duration{2020}{\Ongoing} &
  Earth Sciences representative at the Early Career Academic (ECA) forum,
  \LIV.
  \\
\Year{2015} &
  Chairman of the Election Committee for the deans of the University and the School of
  Geology, \UERJ.
  \\
\Duration{2015}{2017} &
  Faculty Advisor for the Student Chapter of the Socienty of Exploration Geophysicists
  (SEG) at the \UERJ.
\end{EntriesTable}

\subsection{Conference Convener}

\begin{EntriesTable}
%\Year{future} &
\Year{2021} &
  Session: EOS5.3 - The evolving open-science landscape in geosciences: open
  data, software, publications and community initiatives.
  \newline
  Nijzink, RC,
  Drost, N,
  \JFarquharson,
  \AKushnir,
  Pianosi, F,
  Schymanski, S,
  \Me,
  \FWadsworth.
  \newline
  \emph{EGU 2021}, Vienna, Austria.
  \\
  ~ &
  Session: G4.3 - Acquisition and processing of gravity and magnetic field data
  and their integrative interpretation.
  \newline
  \JEbbing, \Carla, \AGuy, \MKaban, \Me.
  \newline
  \emph{EGU 2021}, Vienna, Austria.
  \\
\Year{2019} &
  Townhall: Update and Future Directions of the Open-Source Software Initiative.
  \newline
  \Me, \Lindsey, \Lion, \Rene, \Bane.
  \newline
  \emph{AGU 2019}, San Francisco, USA.
  \\
  ~ &
  Session: NS21A - A Tour of Open-Source Software Packages for the Geosciences.
  \newline
  \Lindsey, \Rene, \Me, \Jens.
  \newline
  \emph{AGU 2019}, San Francisco, USA.
  \\
\Year{2018} &
  Townhall: The role of an open-source software initiative within the AGU.
  \newline
  \Lindsey, \Lion, \Me.
  \newline
  \emph{AGU 2018}, Washington DC, USA.
\end{EntriesTable}

\subsection{Reviewer}

\begin{itemize}
  \item Geophysical Journal International
  \item Journal of Geodesy
  \item Pure and Applied Geophysics
  \item Journal of Applied Geophysics
  \item Geophysical Prospecting
  \item Geophysics
  \item Central European Journal of Geosciences
  \item Computers \& Geosciences
  \item Journal of Open Source Software
\end{itemize}

\subsection{Affiliations}

\begin{EntriesTable}
  \Duration{2020}{\Ongoing} & Royal Astronomical Society
  \\
  \Duration{2014}{\Ongoing} & \href{https://softwareunderground.org}{Software Underground}
  \\
  \Duration{2014}{\Ongoing} & European Geosciences Union
  \\
  \Duration{2010}{\Ongoing} & American Geophysical Union
  \\
  \Duration{2011}{2019} & Society of Exploration Geophysicists
\end{EntriesTable}

\end{document}
