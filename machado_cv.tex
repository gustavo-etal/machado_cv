\documentclass[11pt]{article}

% Full Unicode support for non-ASCII characters
\usepackage[utf8]{inputenc}

% Useful aliases
\newcommand{\UERJ}{Universidade do Estado do Rio de Janeiro}
\newcommand{\UHM}{University of Hawai`i at M\={a}noa}
\newcommand{\NCSU}{North Carolina State University}
\newcommand{\vet}{College of Veterinary Medicine}
\newcommand{\UHEARTH}{Department of Earth Sciences}
\newcommand{\CVM}{Department of Population Health and Pathobiology}
\newcommand{\LIVENV}{School of Environmental Sciences}
\newcommand{\LIV}{University of Liverpool}

% Identifying information
\newcommand{\Title}{Curriculum Vit\ae}
\newcommand{\FirstName}{Gustavo}
\newcommand{\LastName}{Machado}
\newcommand{\Initials}{}
\newcommand{\MyName}{\FirstName\ \LastName}
\newcommand{\Me}{\textbf{\LastName, \Initials}}  % For citations
\newcommand{\Email}{gmachad@ncsu.edu}
\newcommand{\PersonalWebsite}{machado-lab.github.io}
\newcommand{\LabWebsite}{machado-lab.github.io}
\newcommand{\ORCID}{0000-0001-7552-6144}
\newcommand{\Affiliation}{\CVM \\ \vet \\ \NCSU}
\newcommand{\Address}{
  1060 William Moore Drive, Raleigh NC 27607, USA
}

% Names for citing coauthors
\newcommand{\Val}{Barbosa, VCF}
\newcommand{\Bi}{Oliveira Jr, VC}
\newcommand{\Paul}{Wessel, P}
\newcommand{\Joaquim}{Luis, J}
\newcommand{\Remko}{Scharroo, R}
\newcommand{\Florian}{Wobbe, F}
\newcommand{\Walter}{Smith, WHF}
\newcommand{\Dongdong}{Tian, D}
\newcommand{\Bridget}{Smith-Konter, B}
\newcommand{\Eric}{Xu, X}
\newcommand{\David}{Sandwell, DT}
\newcommand{\Carla}{Braitenberg, C}
\newcommand{\Naomi}{Ussami, N}
\newcommand{\Manoel}{D'Agrella-Filho, MS}
\newcommand{\JB}{Silva, JBC}
\newcommand{\Dai}{Sales, DP}
\newcommand{\Figura}{Melo, FF}
\newcommand{\Dio}{Carlos, DU}
\newcommand{\BragaVale}{Braga, MA}
\newcommand{\YLi}{Li, Y}
\newcommand{\Angeli}{Angeli, G}
\newcommand{\Peres}{Peres, G}
\newcommand{\Everton}{Bomfim, EP}
\newcommand{\Eder}{Molina, E}
\newcommand{\Gomes}{Gomes, AAS}
\newcommand{\Santiago}{Soler, SR}
\newcommand{\Agustina}{Pesce, A}
\newcommand{\Gimenez}{Gimenez, ME}
\newcommand{\Kristoffer}{Hallam, KAT}
\newcommand{\Guangdong}{Zhao, G}
\newcommand{\Bo}{Chen, B}
\newcommand{\JLiu}{Liu, J}
\newcommand{\LChen}{Chen, L}
\newcommand{\RGuo}{Guo, R}
\newcommand{\MKaban}{Kaban, MK}
\newcommand{\Lindsey}{Heagy, LJ}
\newcommand{\Lion}{Krischer, L}
\newcommand{\Rene}{Gassmoeller, R}
\newcommand{\Bane}{Sullivan, CB}
\newcommand{\Jens}{Klump, JF}
\newcommand{\LBarba}{Barba, LA}
\newcommand{\JBazan}{Bazan, J}
\newcommand{\JBrown}{Brown, J}
\newcommand{\RGuimera}{Guimera, RV}
\newcommand{\MGymrek}{Gymrek, M}
\newcommand{\AHanna}{Alex Hanna}
\newcommand{\KHuff}{Huff, KD}
\newcommand{\DKatz}{Katz, DS}
\newcommand{\CMadan}{Madan, CR}
\newcommand{\KMoerman}{Moerman, KM}
\newcommand{\KNiemeyer}{Niemeyer, KE}
\newcommand{\JPoulson}{Poulson, JL}
\newcommand{\PPrins}{Prins, P}
\newcommand{\KRam}{Ram, K}
\newcommand{\ARokem}{Rokem, A}
\newcommand{\Arfon}{Smith, AM}
\newcommand{\GThiruvathukal}{Thiruvathukal, GK}
\newcommand{\KThyng}{Thyng, KM}
\newcommand{\BWilson}{Wilson, BE}
\newcommand{\Yehudi}{Yehudi, Y}
\newcommand{\Remi}{Rampin, R}
\newcommand{\Hugo}{van Kemenade, H}
\newcommand{\MattTurk}{Turk, M}
\newcommand{\Shapero}{Shapero, D}
\newcommand{\Anderson}{Banihirwe, A}
\newcommand{\Leeman}{Leeman, J}
\newcommand{\JEbbing}{Ebbing, J}
\newcommand{\AGuy}{Guy, A}
\newcommand{\JFarquharson}{Farquharson, J}
\newcommand{\AKushnir}{Kushnir, A}
\newcommand{\FWadsworth}{Wadsworth, F}
\newcommand{\LPerozzi}{Perozzi, L}
\newcommand{\MWieczorek}{Wieczorek, MA}


% Template configuration
%%%%%%%%%%%%%%%%%%%%%%%%%%%%%%%%%%%%%%%%%%%%%%%%%%%%%%%%%%%%%%%%%%%%%%%%%%%%%%%

% Disable hyphenation
\usepackage[none]{hyphenat}

% Control the font size
\usepackage{anyfontsize}

% Icon fonts (requires using xelatex or luatex)
\usepackage[fixed]{fontawesome5}
\usepackage{academicons}

% Template variables for styling
\newcommand{\TablePad}{\vspace{-0.4cm}}
\newcommand{\SoftwareTitle}[1]{{\bfseries #1}}
\newcommand{\TableTitle}[1]{{\fontsize{12pt}{0}\selectfont \itshape #1}}

% For fancy and multipage tables
\usepackage{tabularx}
\usepackage{ltablex}

% Define a new environment to place all CV entries in a 2-column table.
% Left column are the dates, right column the entries.
\usepackage{environ}
\NewEnviron{EntriesTable}{
\TablePad
\begin{tabularx}{\textwidth}{@{}p{0.12\textwidth}@{\hspace{0.02\textwidth}}p{0.86\textwidth}@{}}
  \BODY
\end{tabularx}
}

% Macros to add links and mark publications
\newcommand{\DOI}[1]{doi:\href{https://doi.org/#1}{#1}}
\newcommand{\DOILink}[1]{\href{https://doi.org/#1}{doi.org/#1}}
\newcommand{\Preprint}[1]{\newline • Preprint: \faFilePdf\ \DOILink{#1}}
\newcommand{\Youtube}[1]{\newline • Recording: \faYoutube\, \href{https://www.youtube.com/watch?v=#1}{youtube.com/watch?v=#1}}
\newcommand{\GitHub}[1]{\newline • Code: \faGithub\ \href{https://github.com/#1}{#1}}
\newcommand{\Role}[1]{\newline • Role: #1}
\newcommand{\Website}[1]{\newline • Website: \href{https://#1}{#1}}
\newcommand{\Slides}[1]{\newline • Slides: \faTv\ \href{https://#1}{#1}}
\newcommand{\SlidesDOI}[1]{\newline • Slides: \faTv\ \DOILink{#1}}
\newcommand{\PosterDOI}[1]{\newline • Poster: \faImage\ \DOILink{#1}}
\newcommand{\OA}{\aiOpenAccess\enspace}
\newcommand{\Invited}{\newline • \textbf{Invited talk}}

% Macros to set the year and duration on the left column
\newcommand{\Duration}[2]{\fontsize{10pt}{0}\selectfont #1 -- #2}
\newcommand{\Year}[1]{\fontsize{10pt}{0}\selectfont #1}
\newcommand{\Ongoing}{present}
%\newcommand{\Ongoing}{$\ast$}
\newcommand{\Future}{future}
\newcommand{\Review}{in review}
\newcommand{\Accepted}{accepted}
\newcommand{\Appointment}[4]{\textbf{#1} \newline #2 \newline #3 \newline #4}

% Define command to insert month name and year as date
\usepackage{datetime}
\newdateformat{monthyear}{\monthname[\THEMONTH], \THEYEAR}

% Set the page margins
\usepackage[a4paper,margin=1.5cm,includehead,headsep=5mm]{geometry}

% To get the total page numbers (\pageref{LastPage})
\usepackage{lastpage}

% No indentation
\setlength\parindent{0cm}

% Increase the line spacing
\renewcommand{\baselinestretch}{1.1}
% and the spacing between rows in tables
\renewcommand{\arraystretch}{1.5}

% Remove space between items in itemize and enumerate
\usepackage{enumitem}
\setlist{nosep}

% Use custom colors
\usepackage[usenames,dvipsnames]{xcolor}

% Set fonts. Requires compilation with xelatex
\usepackage{fontspec}  % required to make older xelatex compile with UTF8

% Configure the font style for sections
\usepackage{sectsty}
\sectionfont{\vspace{0.5cm}\bfseries\fontsize{12pt}{0}\selectfont\uppercase}
\subsectionfont{\vspace{0.2cm}\mdseries\fontsize{12pt}{0}\selectfont\uppercase}

% Set the spacing for sections
%\usepackage{titlesec}
%\titlespacing{\section}{0pt}{0cm}{0.3cm}
%\titlespacing{\subsection}{0pt}{0.3cm}{0.3cm}

% Disable number of sections. Use this instead of "section*" so that the sections still
% appear as PDF bookmarks. Otherwise, would have to add the table of contents entries
% manually.
\makeatletter
\renewcommand{\@seccntformat}[1]{}
\makeatother

% Set fancy headers
\usepackage{fancyhdr}
\pagestyle{fancy}
\fancyhf{}
\chead{
  \fontsize{10pt}{12pt}\selectfont
  \MyName
  \hspace{0.2cm} -- \hspace{0.2cm}
  \Title
  \hspace{0.2cm} -- \hspace{0.2cm}
  \monthyear\today
}
\rhead{\fontsize{10pt}{0}\selectfont \thepage/\pageref*{LastPage}}
\renewcommand{\headrulewidth}{0pt}

% Metadata for the PDF output and control of hyperlinks
\usepackage[colorlinks=true]{hyperref}
\hypersetup{
  pdftitle={\MyName\ - \Title},
  pdfauthor={\MyName},
  linkcolor=blue,
  citecolor=blue,
  filecolor=black,
  urlcolor=MidnightBlue
}
%%%%%%%%%%%%%%%%%%%%%%%%%%%%%%%%%%%%%%%%%%%%%%%%%%%%%%%%%%%%%%%%%%%%%%%%%%%%%%%


\begin{document}

% No header for the first page
\thispagestyle{empty}

%%%%%%%%%%%%%%%%%%%%%%%%%%%%%%%%%%%%%%%%%%%%%%%%%%%%%%%%%%%%%%%%%%%%%%%%%%%%%%%
% HEADER
{\fontsize{22pt}{0}\selectfont\MyName}\\[-0.1cm]
\rule{\textwidth}{0.2pt}
\begin{minipage}[t]{0.595\textwidth}
  \Affiliation
  \\
  \Address
\end{minipage}
\begin{minipage}[t]{0.405\textwidth}
  \begin{flushright}
  Last updated: \monthyear\today
  \\
    ORCID: \href{https://orcid.org/\ORCID}{\ORCID}
    \\
    email: \href{mailto:\Email}{\Email}
    \\
    Lab web site: \href{https://www.\LabWebsite}{\LabWebsite}
  \end{flushright}
\end{minipage}

%%%%%%%%%%%%%%%%%%%%%%%%%%%%%%%%%%%%%%%%%%%%%%%%%%%%%%%%%%%%%%%%%%%%%%%%%%%%%%%
\section{Professional Appointments}

\begin{EntriesTable}
  \Duration{2018}{\Ongoing}  &
  \Appointment{Assistant Professor}{\CVM}{\vet}{\NCSU, USA}
  \\
  \Duration{2016}{2017}  &
  \Appointment{Postdoctoral researcher}{Department of Veterinary Population Medicine }{College of Veterinary Medicine }{University of Minnesota, USA}
  \\
  \Duration{2016} &
  \Appointment{Assistant Professor}{Departamento de Medicina Veterinaria Preventiva}{Faculdade de Medicina Veterinaria}{Universidade Federal do Rio Grande do Sul, Brazil}
\end{EntriesTable}


%%%%%%%%%%%%%%%%%%%%%%%%%%%%%%%%%%%%%%%%%%%%%%%%%%%%%%%%%%%%%%%%%%%%%%%%%%%%%%%
\section{Education}

\begin{EntriesTable}
  \Duration{2013}{2016}  &
  \textbf{PhD in Veterinary Epidemiology}, Universidade Federal do Rio Grande do Sul, Brazil
  \\
  \Duration{2011}{2013}  &
  \textbf{MVSc in Veterinary Epidemiology}, Universidade Federal do Rio Grande do Sul, Brazil
  \\
  \Duration{2003}{2010}  &
  \textbf{DVM Veterinary Medicine}, Universidade Federal de Santa Maria, Brazil
\end{EntriesTable}

%%%%%%%%%%%%%%%%%%%%%%%%%%%%%%%%%%%%%%%%%%%%%%%%%%%%%%%%%%%%%%%%%%%%%%%%%%%%%%%
\section{Funding}
Active
\begin{EntriesTable}
  \Duration{2021}{2022}  &
  \textbf{PI}: \Me
  ``Combining standardized on-farm biosecurity plans with animal movement data in a user-friendly rapid access biosecurity management tool: a multi state study.''. \textbf {Funded by:}
  Animal and Plant Health Inspection Service (USDA-APHIS)-National Animal Disease Preparedness and Response Program (NADPRP)``
  \textit{Amount \$182,301}. Award ID: {1948602}.
  \\
  \Year{2020}  &
  Software Sustainability Institute Fellowship.
  \textit{\LIV}.
  More information:
  \href{https://www.leouieda.com/research/ssi2020.html}{leouieda.com/research/ssi2020.html}
  \\
  \Duration{2019}{2020}  &
  ESRG Research Support:
  ``Geophysical inversion of GRACE satellite time-lapse gravity''.
  Internal fund to support the research visit of PhD student Santiago R.
  Soler.
  \textit{\LIV}.
  \\
  \Duration{2018}{2020}  &
  NSF-EAR: ``The EarthScope/GMT Analysis and Visualization Toolbox''.
  PI: \Paul, \textbf{co-PI}: \Me, co-PI: \Bridget.
  \textit{\UHM}.
  Award ID: \href{https://www.nsf.gov/awardsearch/showAward?AWD_ID=1829371}{1829371}.
  \\
  \Duration{2014}{2018}  &
  QUALITEC/UERJ Grant for training a technician for the Laboratory of
  Exploration Geophysics - \UERJ
\end{EntriesTable}


%%%%%%%%%%%%%%%%%%%%%%%%%%%%%%%%%%%%%%%%%%%%%%%%%%%%%%%%%%%%%%%%%%%%%%%%%%%%%%%
\section{Awards \& Honors}

\begin{EntriesTable}
  \Year{2017}  &
  Brazilian Geophysical Society (SBGf) Award for \textbf{Best PhD Thesis}
  of 2015 -- 2017
  \\
  \Year{2016}  &
  \UERJ, Brazil, School of Geology
  \textbf{Teaching Award} given by the graduating class of 2016
  \\
  \Duration{2011}{2015}  &
  Brazilian Ministry of Education CAPES \textbf{PhD Research Scholarship}
  \\
  \Year{2011}  &
  SEG Near Surface Geophysics Section \textbf{Student Travel Grant} to
  present at the SEG Annual Meeting, San Antornio, TX, USA
  \\
  \Year{2011}  &
  EAGE \textbf{PACE Student Travel Grant} to present at the 73rd EAGE
  Conference \& Exhibition, Vienna, Austria
  \\
  \Duration{2010}{2011}  &
  Brazilian Ministry of Education CAPES \textbf{Masters Research Scholarship}
  \\
  \Year{2008}  &
  Brazilian Geophysical Society (SBGf) \textbf{Undergraduate Research
  Scholarship}
  \\
  \Year{2005}  &
  São Paulo Research Foundation (FAPESP) \textbf{Undergraduate Research
  Scholarship}
\end{EntriesTable}


%%%%%%%%%%%%%%%%%%%%%%%%%%%%%%%%%%%%%%%%%%%%%%%%%%%%%%%%%%%%%%%%%%%%%%%%%%%%%%%
\section{Publications}

\subsection{Preprints}

\begin{EntriesTable}
\Year{2019}  &
  \OA
  \LBarba, \JBazan, \JBrown, \RGuimera, \MGymrek, \AHanna, \Lindsey, \KHuff, \DKatz,
  \CMadan, \KMoerman, \KNiemeyer, \JPoulson, \PPrins, \KRam, \ARokem, \Arfon,
  \GThiruvathukal, \KThyng, \Me, \BWilson, \Yehudi.
  Giving software its due through community-driven review and publication.
  \emph{OSF Preprints}.
  \DOI{10.31219/osf.io/f4vx6}
\end{EntriesTable}

\subsection{Papers}

\begin{EntriesTable}
\Year{2021}  &
  \Santiago, \Me.
  Gradient-boosted equivalent sources.
  \emph{Geophysical Journal International}.
  \DOI{10.1093/gji/ggab297}.
  \Preprint{10.31223/X58G7C}
  \GitHub{compgeolab/eql-gradient-boosted}
  \\
\Year{2020}  &
  \OA
  \Me, \Santiago, \Remi, \Hugo, \MattTurk, \Shapero, \Anderson, \Leeman.
  Pooch: A friend to fetch your data files.
  \emph{Journal of Open Source Software}.
  \DOI{10.21105/joss.01943}.
  \GitHub{fatiando/pooch}
  \\
\Year{2019}  &
  \OA
  \Paul, \Joaquim, \Me, \Remko, \Florian, \Walter, \Dongdong.
  The Generic Mapping Tools, Version 6.
  \emph{Geochemistry, Geophysics, Geosystems}.
  \DOI{10.1029/2019GC008515}.
  \\
  ~ &
  \Santiago, \Agustina, \Gimenez, \Me.
  Gravitational field calculation in spherical coordinates using variable densities in
  depth.
  \emph{Geophysical Journal International}.
  \DOI{10.1093/gji/ggz277}.
  \Preprint{10.31223/osf.io/3548g}
  \GitHub{pinga-lab/tesseroid-variable-density}
  \\
  ~ &
  \Guangdong, \Bo, \Me, \JLiu, \MKaban, \LChen, \RGuo.
  Efficient 3D large-scale forward-modeling and inversion of gravitational fields in
  spherical coordinates with application to lunar mascons.
  \emph{Journal of Geophysical Research: Solid Earth}.
  \DOI{10.1029/2019jb017691}.
  \Preprint{10.31223/osf.io/dzf9j}
  \\
\Year{2018}  &
  \OA
  \Me. Verde: Processing and gridding spatial data using Green's functions.
  \emph{Journal of Open Source Software}.
  \DOI{10.21105/joss.00957}.
  \GitHub{fatiando/verde}
  \\
\Year{2017}  &
  \Me, \Val.
  Fast non-linear gravity inversion in spherical coordinates with application
  to the South American Moho,
  \emph{Geophysical Journal International},
  \DOI{10.1093/gji/ggw390}.
  \Preprint{10.31223/osf.io/9ba4m}
  \GitHub{pinga-lab/paper-moho-inversion-tesseroids}
  \\
\Year{2016}  &
  \Me, \Val, \Carla.
  Tesseroids: forward modeling gravitational fields in spherical coordinates,
  \emph{Geophysics},
  \DOI{10.1190/geo2015-0204.1}.
  \GitHub{pinga-lab/paper-tesseroids}
  \\
  ~ &
  \Dio, \Me, \Val.
  How two gravity-gradient inversion methods can be used to reveal different
  geologic features of ore deposit - A case study from the Quadrilátero
  Ferrífero (Brazil),
  \emph{Journal of Applied Geophysics},
  \DOI{10.1016/j.jappgeo.2016.04.011}.
  \\
\Year{2015}  &
  \OA
  \Bi, \Dai, \Val, \Me.
  Estimation of the total magnetization direction of approximately spherical
  bodies,
  \emph{Nonlinear Processes in Geophysics},
  \DOI{10.5194/npg-22-215-2015}.
  \GitHub{pinga-lab/Total-magnetization-of-spherical-bodies}
  \\
\Year{2014}  &
  \Dio, \Me, \Val.
  Imaging iron ore from the Quadrilátero Ferrífero (Brazil) using geophysical
  inversion and drill hole data,
  \emph{Ore Geology Reviews},
  \DOI{10.1016/j.oregeorev.2014.02.011}.
  \\
\Year{2013}  &
  \Figura, \Val, \Me, \Bi, \JB.
  Estimating the nature and the horizontal and vertical positions of 3D
  magnetic sources using Euler deconvolution,
  \emph{Geophysics},
  \DOI{10.1190/geo2012-0515.1}.
  \\
  ~ &
  \Bi, \Val, \Me.
  Polynomial equivalent layer,
  \emph{Geophysics},
  \DOI{10.1190/geo2012-0196.1}.
  \\
\Year{2012}  &
  \Me, \Val.
  Robust 3D gravity gradient inversion by planting anomalous densities,
  \emph{Geophysics},
  \DOI{10.1190/geo2011-0388.1}.
  \GitHub{pinga-lab/paper-planting-densities}
\end{EntriesTable}


\subsection{Conference proceedings (peer-reviewed)}

\begin{EntriesTable}
\Year{2014}  &
  \Figura, \Val, \Me, \Bi, \JB.
  A Single Euler Solution Per Anomaly,
  \emph{76th EAGE Conference and Exhibition 2014},
  \DOI{10.3997/2214-4609.20140891}.
  \\
\Year{2013}  &
  \Me, \Bi, \Val.
  Modeling the Earth with Fatiando a Terra,
  \emph{Proceedings of the 12th Python in Science Conference}.
  \DOI{10.25080/Majora-8b375195-010}.
  \\
\Year{2012}  &
  \Me, \Val.
  Use of the ``shape-of-anomaly'' data misfit in 3D inversion by planting
  anomalous densities,
  \emph{SEG Technical Program Expanded Abstracts},
  \DOI{10.1190/segam2012-0383.1}.
  \\
  ~ &
  \Dio, \Me, \YLi, \Val, \BragaVale, \Angeli, \Peres.
  Iron ore interpretation using gravity-gradient inversions in the Carajás, Brazil.
  \emph{SEG Technical Program Expanded Abstracts},
  \DOI{10.1190/segam2012-0525.1}.
  \\
\Year{2011}  &
  \Me, \Everton, \Carla, \Eder.
  Optimal forward calculation method of the Marussi tensor due to a geologic
  structure at GOCE height,
  \emph{Proceedings of the 4th International GOCE User Workshop}.
  \\
  ~ &
  \Me, \Val.
  Robust 3D gravity gradient inversion by planting anomalous densities,
  \emph{SEG Technical Program Expanded Abstracts},
  \DOI{10.1190/1.3628201}.
  \\
  ~ &
  \Me, \Val.
  3D gravity inversion by planting anomalous densities.
  \emph{12th International Congress of the Brazilian Geophysical Society},
  \DOI{10.1190/sbgf2011-179}.
  \\
  ~ &
  \Me, \Val.
  3D gravity gradient inversion by planting density anomalies.
  \emph{73th EAGE Conference and Exhibition incorporating SPE EUROPEC},
  \DOI{10.3997/2214-4609.20149567}.
  \\
  ~ &
  \Dio, \Me, \Val, \BragaVale, \Gomes.
  In-depth imaging of an iron orebody from Quadrilatero Ferrifero using 3D
  gravity gradient inversion,
  \emph{SEG Technical Program Expanded Abstracts},
  \DOI{10.1190/1.3628219}.
  \\
  ~ &
  \Dio, \Val, \Me, \BragaVale.
  Inversão de Dados de Aerogradiometria Gravimétrica 3D-FTG Aplicada a
  Exploração Mineral na Região do Quadrilátero Ferrífero,
  \emph{12th International Congress of the Brazilian Geophysical Society},
  \DOI{10.1190/sbgf2011-243}.
\end{EntriesTable}

\subsection{Open Datasets}

\begin{EntriesTable}
\Year{2020}  &
  \Me.
  Ground gravity data compilation for Australia filtered by survey quality
  and packaged in CF-compliant netCDF (derived from the
  Geoscience Australia compilation by \href{https://doi.org/10.26186/5c1987fa17078}{Wynne (2018)}).
  \DOI{10.6084/m9.figshare.13643837}
  \GitHub{compgeolab/australia-gravity-data}
  \\
\Year{2017}  &
  \Me, \Val.
  A gravity-derived Moho model for South America: source code, data, and
  model results from ``Fast non-linear gravity inversion in spherical
  coordinates with application to the South American Moho''.
  \DOI{10.6084/m9.figshare.3987267}
  \GitHub{pinga-lab/paper-moho-inversion-tesseroids}
\end{EntriesTable}


\subsection{Open-source Software}

\begin{EntriesTable}
  \Duration{2017}{\Ongoing} &
  \textbf{PyGMT}
  \newline
  A Python interface for the Generic Mapping Tools
  \Role{Creator and core developer}
  \GitHub{GenericMappingTools/pygmt}
  \Website{www.pygmt.org}
  \\
  \Duration{2017}{\Ongoing} &
  \textbf{The Generic Mapping Tools (GMT)}
  \newline
  A data processing and mapping toolbox for the Earth, Ocean, and Planetary Science
  \Role{Core team and community management}
  \GitHub{GenericMappingTools/gmt}
  \Website{www.generic-mapping-tools.org}
  \\
  \Duration{2010}{\Ongoing} &
  \textbf{Fatiando a Terra}
  \newline
  Python tools for geophysical data processing, forward modeling, and inversion
  \Role{Creator, main developer, project leadership}
  \GitHub{fatiando}
  \Website{www.fatiando.org}
  \\
  \Duration{2009}{2016} &
  \textbf{Tesseroids}
  \newline
  Forward modeling of gravitational fields in spherical coordinates
  \Role{Creator and sole developer}
  \GitHub{leouieda/tesseroids}
  \Website{www.tesseroids.org}
\end{EntriesTable}


%%%%%%%%%%%%%%%%%%%%%%%%%%%%%%%%%%%%%%%%%%%%%%%%%%%%%%%%%%%%%%%%%%%%%%%%%%%%%%%
\section{Teaching}

\subsection{Undergraduate}

\begin{EntriesTable}
  \Duration{2020}{\Ongoing}  &
  ENVS101/106: Study Skills and GIS (tutorial)
  \newline
  Leading small group tutorials and a Python programming workshop
  \newline
  \textit{\LIV}
  \\
  \Duration{2020}{\Ongoing}  &
  ENVS398: Global Geophysics and Geodynamics
  \newline
  Teaching lithosphere dynamics (50\% of module)
  \newline
  Module coordinator from 2021
  \newline
  \textit{\LIV}
  \\
  \Duration{2020}{\Ongoing}  &
  ENVS258: Environmental Geophysics
  \newline
  Teaching remote sensing, gravimetry, and Python programming
  ($\sim$50\% of module)
  \newline
  \textit{\LIV}
  \\
  \Duration{2019}{\Ongoing}  &
  ENVS363: Geophysical Exploration Techniques (field)
  \newline
  Part of the teaching team for geophysical field methods
  \newline
  \textit{\LIV}
  \\
  \Duration{2019}{\Ongoing}  &
  ENVS123: Introduction to Geoscience and Earth History
  \newline
  Lectures on: Earth's internal structure; gravity and isostasy
  \newline
  \textit{\LIV}
  \\
  \Duration{2014}{2016}  &
  Special Mathematics I: Introduction to Programming and Numerical Analysis
  \newline
  \textit{\UERJ}
  \GitHub{mat-esp/about}
  \\
  \Duration{2014}{2016}  &
  Geophysics I: Gravity and magnetic methods
  \newline
  \textit{\UERJ}
  \GitHub{leouieda/geofisica1}
  \\
  \Duration{2014}{2016}  &
  Geophysics II: Exploration Seismology
  \newline
  \textit{\UERJ}
  \GitHub{leouieda/geofisica2}
  \\
  \Year{2015}  &
  Introduction to Geology
  \newline
  \textit{\UERJ}
\end{EntriesTable}


\subsection{Workshops \& Short Courses}

\begin{EntriesTable}
%\Year{future}  &
\Year{2021} &
  The Generic Mapping Tools for Geodesy
  \newline
  \textit{UNAVCO} (online)
  \GitHub{GenericMappingTools/2021-unavco-course}
  \\
\Year{2020} &
  Let's build a geophysical inversion with Python
  \newline
  \textit{IRTG-2379 Graduate School: Modern Inverse Problems}
  \newline
  \textit{RWTH Aachen University} (online)
  \GitHub{compgeolab/2020-aachen-inverse-problems}
  \\
  ~ &
  The Generic Mapping Tools for Geodesy
  \newline
  \textit{UNAVCO} (online)
  \Youtube{EQgxDmCXvj4}
  \GitHub{GenericMappingTools/2020-unavco-course}
  \\
  ~  &
  From scattered data to gridded products using Verde
  \newline
  \textit{Transform 2020} (online)
  \Youtube{-xZdNdvzm3E}
  \GitHub{fatiando/transform2020}
  \\
\Year{2019}  &
  Best Practices for Developing and Sustaining Your Open-Source Research Software
  \newline
  \textit{AGU Fall Meeting 2019}
  \GitHub{agu-ossi/2019-agu-oss}
  \\
  ~  &
  Become a Generic Mapping Tools Contributor Even If You Can't Code
  \newline
  \textit{AGU Fall Meeting 2019}
  \\
  ~  &
  The Generic Mapping Tools for Geodesy
  \newline
  \textit{Scripps Institution of Oceanography} and \textit{UNAVCO}
  \Youtube{uPUt4\_kd6m8}
  \GitHub{GenericMappingTools/2019-unavco-course}
  \\
  ~  &
  Introduction to Python Workshop (Earth Sciences REU program)
  \newline
  \textit{Department of Geology and Geophysics, \UHM}
  \GitHub{leouieda/2019-06-reu-python}
  \\
\Year{2018}  &
  Best Practices for Modern Open-Source Research Codes
  \newline
  \textit{AGU Fall Meeting 2018}
  \GitHub{agu-ossi/2018-agu-oss}
  \\
  ~  &
  Git and Github: What are their uses? Are they worth the effort? Let's find out!
  \newline
  \textit{ASPRS UHM Student Chapter, \UHM}
  \\
\Year{2017}  &
  Introduction to Python
  \newline
  \textit{Department of Geology and Geophysics, \UHM}
  \GitHub{leouieda/python-hawaii-2017}
  \\
\Year{2016}  &
  Python for Geologists (SAGEO)
  \newline
  \textit{Faculdade de Geologia, \UERJ}
  \GitHub{leouieda/python-geologia-2016}
  \\
  ~  &
  Python for Earth Scientists (IAG Summer School)
  \newline
  \textit{Departamento de Geofísica, Universidade de São Paulo}
  \GitHub{leouieda/verao2016}
  \\
\Year{2014}  &
  Introduction to Geophysical Inversion
  \newline
  \textit{Instituto de Geociências, Universidade de Brasília}
  \GitHub{pinga-lab/inversao-unb-2014}
  \\
\Year{2011}  &
  Introduction to Geophysical Inversion (IAG Summer School)
  \newline
  \textit{Departamento de Geofísica, Universidade de São Paulo}
  \GitHub{pinga-lab/inversao-iag-2012}
\end{EntriesTable}


%%%%%%%%%%%%%%%%%%%%%%%%%%%%%%%%%%%%%%%%%%%%%%%%%%%%%%%%%%%%%%%%%%%%%%%%%%%%%%%
\section{Student supervision}

\subsection{P\lowercase{h}D}

\begin{EntriesTable}
\Duration{2017}{\Ongoing}  &
  Santiago R. Soler (co-Advising)
  \newline
  Universidad Nacional de San Juan, Argentina.
  \newline
  Advisor: Mario E. Gimenez
\end{EntriesTable}

\subsection{Master's}

\begin{EntriesTable}
\Duration{2020}{2021}  &
  Aidan Hernaman
  \newline
  \LIV, UK.
\end{EntriesTable}

\subsection{Undergraduate}

\begin{EntriesTable}
\Duration{2020}{2021}  &
  Majed M.A. Abura, Ali A.A. Alhazmi, Daniel P. Gilbert, and Mustafa M.M.
  Alordowny
  \newline
  \LIV, UK.
  \\
\Duration{2019}{2020}  &
  Lottie Cooper, Steven Heer, Charles Thomson, and Alexander Borges
  \newline
  \LIV, UK.
  \\
\Duration{2015}{2017}  &
  Vinicius V. Riguete
  \newline
  \UERJ, Brazil.
  \\
\end{EntriesTable}


%%%%%%%%%%%%%%%%%%%%%%%%%%%%%%%%%%%%%%%%%%%%%%%%%%%%%%%%%%%%%%%%%%%%%%%%%%%%%%%
\section{Presentations}

\begin{EntriesTable}
\Year{2021}  &
  \Me.
  Academia e software livre: Desafios e oportunidades no Brasil e no exterior,
  \emph{National Observatory's SEG and EAGE Student Chapter},
  Rio de Janeiro, Brazil.
  \Invited{}
  \GitHub{leouieda/2021-07-22-on}
  \Youtube{r2x-DN6laj8}
  \\
  ~ &
  \Me, \Santiago, \Agustina.
  Open-science for gravimetry: tools, challenges, and opportunities,
  \emph{GFZ Helmholtz Centre Potsdam},
  Germany.
  \Invited{}
  \GitHub{leouieda/2021-06-22-gfz}
  \SlidesDOI{10.6084/m9.figshare.14838477}
  \Youtube{z-5dvWfB\_SM}
  \\
  ~ &
  \Me, \Santiago, \Agustina.
  Fatiando a Terra: Open-source tools for geophysics,
  \emph{Geophysical Society of Houston},
  Houston, USA.
  \Invited{}
  \GitHub{fatiando/2021-gsh}
  \\
  ~ &
  \Me, \Santiago, \Agustina, \LPerozzi, \MWieczorek.
  Harmonica and Boule: Modern Python tools for geophysical gravimetry,
  \emph{EGU 2021},
  Online.
  \DOI{10.5194/egusphere-egu21-8291}.
  \GitHub{fatiando/egu2021}
  \\
\Year{2020}  &
  \Me.
  Geophysical research powered by open-source,
  \emph{Christian Albrechts Universität zu Kiel},
  Kiel, Germany.
  \Invited
  \Slides{www.leouieda.com/2020-07-01-kiel}
  \\
  ~ &
  \Me.
  Geophysical research powered by open-source,
  \emph{Departamento de Geofísica, IAG, Universidade de São Paulo},
  São Paulo, Brazil.
  \Invited
  \Youtube{VqI8BX1Yg54}
  \Slides{www.leouieda.com/2020-06-18-usp}
  \\
  ~ &
  \Me.
  Geophysical research powered by open-source,
  \emph{Technische Universität Bergakademie Freiberg},
  Freiberg, Germany.
  \Invited
  \Slides{www.leouieda.com/2020-06-04-freiberg}
  \\
  ~ &
  \Me.
  Geophysical research powered by open-source,
  \emph{Geographic Data Science Lab, University of Liverpool},
  Liverpool, UK.
  \Invited
  \Slides{www.leouieda.com/liverpool-gdsl-2020}
  \\
  ~ &
  \Me, \Santiago.
  Evaluating the accuracy of equivalent-source predictions using
  cross-validation,
  \emph{EGU 2020},
  Vienna, Austria.
  \DOI{10.5194/egusphere-egu2020-15729}.
  \SlidesDOI{10.6084/m9.figshare.12245372}
  \\
\Year{2019}  &
  \Me, \Paul.
  PyGMT: Accessing the Generic Mapping Tools from Python,
  \emph{AGU 2019},
  San Francisco, USA.
  \PosterDOI{10.6084/m9.figshare.11320280}
  \\
  ~ &
  \Me.
  Building the foundations for open-source geophysics,
  \emph{\CVM, \LIV},
  UK.
  \SlidesDOI{10.6084/m9.figshare.10255832}
  \\
\Year{2018}  &
  \Me, \Eric, \Paul, \David.
  Coupled Interpolation of Three-component GPS Velocities,
  \emph{AGU 2018},
  Washington DC, USA.
  \PosterDOI{10.6084/m9.figshare.7440683}
  \\
  ~ &
  \Me.
  Machine Learning Lessons for Geophysics,
  \emph{Department of Earth Sciences, \UHM},
  Honolulu, USA.
  \SlidesDOI{10.6084/m9.figshare.7203344}
  \\
  ~ &
  \Me, \Paul.
  Building an object-oriented Python interface for the Generic Mapping Tools,
  \emph{Scipy 2018},
  Austin, USA.
  \Youtube{6wMtfZXfTRM}
  \SlidesDOI{10.6084/m9.figshare.6814052}
  \\
  ~ &
  \Me, \David, \Paul.
  Joint Interpolation of 3-component GPS Velocities Constrained by
  Elasticity,
  \emph{AOGS $15^{th}$ Annual Meeting},
  Honolulu, USA.
  \SlidesDOI{10.6084/m9.figshare.6387467}
  \\
  ~ &
  \Me, \Paul.
  Integrating the Generic Mapping Tools with the Scientific Python Ecosystem,
  \emph{AOGS $15^{th}$ Annual Meeting},
  Honolulu, USA.
  \PosterDOI{10.6084/m9.figshare.6399944}
  \\
\Year{2017}  &
  \Me, \Paul.
  Nurturing reliable and robust open-source scientific software,
  \emph{AGU Fall Meeting 2017},
  New Orleans, USA.
  \Invited
  \Youtube{0GO4ZZ5Ry6M}
  \\
  ~  &
  \Me, \Paul.
  A modern Python interface for the Generic Mapping Tools,
  \emph{AGU Fall Meeting 2017},
  New Orleans, USA.
  \PosterDOI{10.6084/m9.figshare.5662411}
  \\
  ~  &
  \Me, \Paul.
  Bringing the Generic Mapping Tools to Python,
  \emph{Scipy 2017},
  Austin, USA.
  \Youtube{93M4How7R24}
  \SlidesDOI{10.6084/m9.figshare.7635833}
  \\
  ~ &
  \Me.
  Inverting gravity to map the Moho: A new method and the open source
  software that made it possible,
  \emph{Department of Geology and Geophysics, \UHM},
  Honolulu, USA.
  \SlidesDOI{10.6084/m9.figshare.4779766}
  \\
\Year{2016}  &
  \Me.
  Fatiando a Terra: construindo uma base para ensino e pesquisa de geofísica,
  \emph{Observatório Nacional},
  Rio de Janeiro, Brazil.
  \Invited
  \SlidesDOI{10.6084/m9.figshare.1381870}
  \\
\Year{2015}  &
  \Me.
  Fatiando a Terra: construindo uma base para ensino e pesquisa de geofísica,
  \emph{Universidade de São Paulo},
  São Paulo, Brazil.
  \Invited
  \SlidesDOI{10.6084/m9.figshare.1381870}
  \\
\Year{2014}  &
  \Me, \Bi, \Val.
  Using Fatiando a Terra to solve inverse problems in geophysics,
  \emph{Scipy 2014},
  Austin, USA.
  \PosterDOI{10.6084/m9.figshare.1089987}
  \\
  ~ &
  \Me, \Val.
  Gravity inversion in spherical coordinates using tesseroids,
  \emph{EGU General Assembly 2014},
  Vienna, Austria.
  \SlidesDOI{10.6084/m9.figshare.1155457}
  \\
\Year{2013}  &
  \Me, \Bi, \Val.
  Modeling the Earth with Fatiando a Terra,
  \emph{Scipy 2013},
  Austin, USA.
  \DOI{10.25080/Majora-8b375195-010}.
  \Youtube{Ec38h1oB8cc}
  \Slides{www.leouieda.com/scipy2013/?theme=night}
  \\
  ~ &
  \Me, \Val.
  3D magnetic inversion by planting anomalous densities,
  \emph{AGU Meeting of the Americas},
  Cancun, Mexico.
  \SlidesDOI{10.6084/m9.figshare.703651}
  \\
\Year{2012}  &
  \Dio, \Me, \YLi, \Val, \BragaVale, \Angeli, \Peres.
  Iron ore interpretation using gravity-gradient inversions in the Carajás,
  Brazil,
  \emph{SEG Annual Meeting 2012},
  Las Vegas, USA.
  \DOI{10.1190/segam2012-0525.1}.
  \SlidesDOI{10.6084/m9.figshare.156865}
  \\
  ~ &
  \Me, \Val.
  Use of the ``shape-of-anomaly'' data misfit in 3D inversion by planting
  anomalous densities,
  \emph{SEG Annual Meeting 2012},
  Las Vegas, USA.
  \DOI{10.1190/segam2012-0383.1}.
  \SlidesDOI{10.6084/m9.figshare.156864}
  \\
  ~ &
  \Me, \Val.
  Rapid 3D inversion of gravity and gravity gradient data to test geologic
  hypotheses,
  \emph{International Symposium on Gravity, Geoid and Height Systems},
  Venice, Italy.
  \SlidesDOI{10.6084/m9.figshare.156859}
  \\
\Year{2011}  &
  \Me, \Val.
  Robust 3D gravity gradient inversion by planting anomalous densities,
  \emph{SEG Annual Meeting 2011},
  San Antonio, USA.
  \DOI{10.1190/1.3628201}.
  \SlidesDOI{10.6084/m9.figshare.156863}
  \\
  ~ &
  \Me, \Val.
  3D gravity inversion by planting anomalous densities,
  \emph{Internation Congress of the Brazilian Geophysical Society},
  Rio de Janeiro, Brazil.
  \DOI{10.1190/sbgf2011-179}.
  \SlidesDOI{10.6084/m9.figshare.156861}
  \\
  ~ &
  \Me, \Everton, \Carla, \Eder.
  Optimal forward calculation method of the Marussi tensor due to a geologic
  structure at GOCE height,
  \emph{4th International GOCE User Workshop},
  Munich, Germany.
  \PosterDOI{10.6084/m9.figshare.92624}
  \\
  ~ &
  \Me, \Val.
  3D gravity gradient inversion by planting density anomalies,
  \emph{73th EAGE Conference and Exhibition incorporating SPE EUROPEC},
  Vienna, Austria.
  \DOI{10.3997/2214-4609.20149567}.
  \PosterDOI{10.6084/m9.figshare.91511}
  \\
\Year{2010}  &
  \Me, \Naomi, \Carla.
  Computation of the gravity gradient tensor due to topographic masses using
  tesseroids,
  \emph{AGU Meeting of the Americas},
  Foz do Iguaçu, Brazil.
  \SlidesDOI{10.6084/m9.figshare.156858}
  \\
\Year{2008}  &
  \Me, \Naomi.
  Utilização de tesseróides na modelagem de dados de gradiometria
  gravimétrica,
  \emph{XIII Simpósio de Iniciação Científica do IAG-USP},
  São Paulo, Brazil.
  \PosterDOI{10.6084/m9.figshare.4779760}
  \\
\Year{2006}  &
  \Me, \Manoel.
  Paleomagnetismo e mineralogia magnética dos diques cambrianos de Maravilhas
  e Prata (PB),
  \emph{XI Simpósio de Iniciação Científica do IAG/USP},
  São Paulo, Brazil.
  \PosterDOI{10.6084/m9.figshare.4779769}
\end{EntriesTable}


%%%%%%%%%%%%%%%%%%%%%%%%%%%%%%%%%%%%%%%%%%%%%%%%%%%%%%%%%%%%%%%%%%%%%%%%%%%%%%%
\section{Outreach}

I maintain a blog about my research, geoscience, and programming at
\href{https://www.leouieda.com/blog}{leouieda.com/blog}
\\

\begin{EntriesTable}
\Year{2018}  &
  Interviewed by the geoscience podcast \textit{Don't Panic Geocast}, episode 166
  \textit{``You are headed to a warm and sunny place''}:
  \href{http://www.dontpanicgeocast.com/?p=638}{dontpanicgeocast.com/?p=638}
  \\
\Year{2017}  &
  Volunteer for the \textit{Hour of Code} at Salt Lake Elementary School, Honolulu,
  USA.
  \\
\Year{2016}  &
  Interviewed by the geoscience podcast \textit{Undersampled Radio}, episode
  \textit{``Open Sourcery''}:
  \href{https://undersampledrad.io/home/2016/7/open-sourcery}{undersampledrad.io/home/2016/7/open-sourcery}
\end{EntriesTable}

Geophysical tutorials for the SEG publication \textit{The Leading Edge}:
\\

\begin{EntriesTable}
\Year{2017}  &
  \OA
  \Me.
  Step-by-step NMO correction,
  \emph{The Leading Edge},
  \DOI{10.1190/tle36020179.1}.
  \GitHub{pinga-lab/nmo-tutorial}
  \\
\Year{2014}  &
  \OA
  \Me, \Bi, \Val.
  Geophysical tutorial: Euler deconvolution of potential-field data,
  \emph{The Leading Edge},
  \DOI{10.1190/tle33040448.1}.
  \GitHub{pinga-lab/paper-tle-euler-tutorial}
\end{EntriesTable}


%%%%%%%%%%%%%%%%%%%%%%%%%%%%%%%%%%%%%%%%%%%%%%%%%%%%%%%%%%%%%%%%%%%%%%%%%%%%%%%
\section{Academic Service \& Affiliations}

\subsection{Editor}

\begin{EntriesTable}
  \Duration{2019}{\Ongoing} & Topic editor for the \textit{Journal of Open Source Software}
\end{EntriesTable}

\subsection{Community service}

\begin{EntriesTable}
  \Duration{2019}{\Ongoing} & EarthArXiv Advisory Council
\end{EntriesTable}

\subsection{Committees}

\begin{EntriesTable}
\Duration{2020}{\Ongoing} &
  Department committee for web presence (website, social media, etc),
  \LIV.
  \\
\Duration{2020}{\Ongoing} &
  Earth Sciences representative at the Early Career Academic (ECA) forum,
  \LIV.
  \\
\Year{2015} &
  Chairman of the Election Committee for the deans of the University and the School of
  Geology, \UERJ.
  \\
\Duration{2015}{2017} &
  Faculty Advisor for the Student Chapter of the Socienty of Exploration Geophysicists
  (SEG) at the \UERJ.
\end{EntriesTable}

\subsection{Conference Convener}

\begin{EntriesTable}
%\Year{future} &
\Year{2021} &
  Session: EOS5.3 - The evolving open-science landscape in geosciences: open
  data, software, publications and community initiatives.
  \newline
  Nijzink, RC,
  Drost, N,
  \JFarquharson,
  \AKushnir,
  Pianosi, F,
  Schymanski, S,
  \Me,
  \FWadsworth.
  \newline
  \emph{EGU 2021}, Vienna, Austria.
  \\
  ~ &
  Session: G4.3 - Acquisition and processing of gravity and magnetic field data
  and their integrative interpretation.
  \newline
  \JEbbing, \Carla, \AGuy, \MKaban, \Me.
  \newline
  \emph{EGU 2021}, Vienna, Austria.
  \\
\Year{2019} &
  Townhall: Update and Future Directions of the Open-Source Software Initiative.
  \newline
  \Me, \Lindsey, \Lion, \Rene, \Bane.
  \newline
  \emph{AGU 2019}, San Francisco, USA.
  \\
  ~ &
  Session: NS21A - A Tour of Open-Source Software Packages for the Geosciences.
  \newline
  \Lindsey, \Rene, \Me, \Jens.
  \newline
  \emph{AGU 2019}, San Francisco, USA.
  \\
\Year{2018} &
  Townhall: The role of an open-source software initiative within the AGU.
  \newline
  \Lindsey, \Lion, \Me.
  \newline
  \emph{AGU 2018}, Washington DC, USA.
\end{EntriesTable}

\subsection{Reviewer}

\begin{itemize}
  \item Geophysical Journal International
  \item Journal of Geodesy
  \item Pure and Applied Geophysics
  \item Journal of Applied Geophysics
  \item Geophysical Prospecting
  \item Geophysics
  \item Central European Journal of Geosciences
  \item Computers \& Geosciences
  \item Journal of Open Source Software
\end{itemize}

\subsection{Affiliations}

\begin{EntriesTable}
  \Duration{2020}{\Ongoing} & Royal Astronomical Society
  \\
  \Duration{2014}{\Ongoing} & \href{https://softwareunderground.org}{Software Underground}
  \\
  \Duration{2014}{\Ongoing} & European Geosciences Union
  \\
  \Duration{2010}{\Ongoing} & American Geophysical Union
  \\
  \Duration{2011}{2019} & Society of Exploration Geophysicists
\end{EntriesTable}

\section{Languages}

\TablePad
\begin{tabularx}{\textwidth}{@{}p{0.15\textwidth} p{0.85\textwidth}@{}}
  Portuguese & Native
  \\
  English & IELTS: CEFR Level C2 (mastery or proficiency) obtained in 2019
\end{tabularx}

\end{document}
